\chapter{Programm-Arbeitsweise}\label{chap:background}

\section{Automatische Kundeneigenschaften"=Zuordnung}\label{sec:clientautocompletion}

Folgende Eigenschaften können automatisiert zugeordnet werden: Bundesland, Koordinaten und SLP.

Um dies zu Bewerkstelligen, gibt es Tabellen mit regulären Ausdrücken. Das sind Vorschriften, mit denen bestimmte andere Werte des Kunden geprüft werden. Bei Erfolg wird die hinterlegte Eigenschaft gesetzt. Das ist die automatische Kundeneigenschaften"=Zuordnung.

Im speziellen gibt es die Tabelle "`PLZBundeslandZuordnungen"', mit dessen Hilfe das Programm die PLZ eines Kunden überprüft und ihm ggf. ein Bundesland zuordnet.

Ist dieser Versuch erfolglos, wird die Tabelle "`OrtBundeslandZuordnungen"' verwendet, um dasselbe anhand des Ortsnamen zu versuchen. Der Ortsname wird dafür aus dem Vertragsnamen abgeleitet.

Daneben gibt es die Tabelle "`PLZKoordinatenZuordnungen"', mit dessen Hilfe das Programm die ungefähren geographischen Koordinaten eines Kunden anhand der PLZ zuordnet.

Weiterhin gibt es die Tabelle VertragsnameSLPZuordnungen, mit dessen Hilfe das Programm anhand des Vertragsnamen ein SLP auswählen kann.

In diesen 4 Tabellen entscheidet die Spalte "`Priorität"' über die Priorität dieser Tests. Sind mehrere Tests erfolgreich, wird die Zuordnung mit der niedrigsten Prioritätszahl gewählt.

Zur Syntax der regulären Audrücke mit der verwendeten Engine siehe \url{http://qt.nokia.com/doc/4.5/qregexp.html}.

\section{Mathematik des Ausrollens}\label{sec:rollout_mathematics}

Zunächst werden alle für einen Kunden wichtigen Parameter ermittelt: Die möglichen Koeffizienten für die Sigmoid"=Funktion, die Wochentagsfaktoren und alle zutreffenden Feiertage und Tagestemperaturen für dieses und die 4 vorhergehenden Jahre. Siehe auch \vref{sec:holidays} und besonders \vref{sec:temperatures}.

Einige Werte werden entsprechend der Praxisinformation P2007/13 des BGW \cite{bgwvkup200713} gerundet.

Dann wird der Kundenwert ermittelt, welcher am Ende auf eine ganze Zahl gerundet wird. Dafür werden die Gesamtmengen der letzten 3 Jahre vor dem Ausrolljahr verwendet, sofern explizite Temperaturwerte vorliegen. Der Kundenwert ist die Summe der Gesamtmengen des betrachteten Zeitraumes dividiert durch die Summe der errechneten Tagesverbrauchszahlen, mit Hilfe der Parameter für jeden Tag eines Jahres.

Die Tagesverbrauchszahlen sind das Produkt aus dem passenden Wochenatagsfaktor (bei Feiertagen der Faktors für Sonntag) und der Sigmoidfunktion in Abhängigkeit der passenden Koeffizienten und dem arithmetischen Mittel der Tagesmitteltemperatur der letzten 4 Tage.

Die Sigmoidfunktion:
\begin{equation*}
Sigmoidfunktion = \frac{A}{(1 + \frac{B}{Temperatur-40\degree C}^C)} + D
\end{equation*}

Anschließend werden die Tagesverbrauchszahlen für jeden Tag das auszurollende Jahres ermittelt und mit dem Kundenwert multipliziert, um den Tagesverbrauch in kWh zu berechnen. Hierfür wird auf allgemeine Temperaturwerte ausgewichen, falls für das auszurollende Jahr noch keine expliziten Temperaturwerte vorliegen.
