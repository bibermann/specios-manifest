\chapter{Konzept}\label{chap:concept}

Dieses Kapital wird eine Lösung skizzieren, welche die in \vref{sec:consequences} dargelegten Anforderungen erfüllt.

Ich nenne das System \myindexednote{Specios}\textit{Specios, la specio sistemo}\footnote{"`Specios, la specio sistemo"' (\index{Esperanto}epo.) bedeutet "`Specios, das Arten"=System"', wobei \textit{Art} im Sinne von \textit{Spezies} zu verstehen ist, und \textit{System} u.\,A. im politischen Sinne. Also ein Verwaltungssystem für jede Art von Lebewesen. Wenn man möchte, kann man \textit{Specios} selbst auch ähnlich eines rekursiven Akronyms betrachten, es steht für "`\textbf{Specio} \textbf{s}ystemo por \textbf{s}inkonservo, \textbf{p}rospero, \textbf{e}galrajteco kaj la \textbf{c}elado al \textbf{i}dentigado de \textbf{o}bskureco"' (\index{Esperanto}epo. Arten"=System zur Selbsterhaltung, Wohlstand, Gleichberechtigung und dem Bestreben nach der Identifizierung der Unbekanntheit). Zu beachten ist, dass "`specios"' nicht esperanto ist, bezeichnet also auch nicht die Mehrzahl von "`specio"', dies wäre "`specioj"'.}.

\section{Grundlage}\label{sec:basis}

Das Projekt ist weder speziell für den Menschen konzipiert, stellt ihn also auch nicht unbegründet in den Mittelpunkt, noch ist es explizit für den Planeten Erde entwickelt.

\subsection{Ziel}\label{sec:basis/aim}

Die Frage nach dem eigentlichen Ziel ist die tiefgreifendste und zugleich schwierigste. Daher wird nun versucht, dieses so wissenschaftlich wie möglich herzuleiten. Moral soll hier keine Rolle spielen, da sie kulturabhängig und meist auf religiös verankerte Axiome fundiert ist.

\subsubsection{Feststellungen}

Zunächst muss man feststellen, dass es keinen \myindexednote{Gott}göttlichen Grund für die Existenz von Lebewesen gibt und auch kein höheres Ziel, wie z.B. das Leben selbst oder den Wohlstand im engeren Sinne. Man muss verstehen, dass wir lediglich die zufällige Folge einer physikalischen Kettenreaktion sind. Angefangen beim Urknall, der zwar noch ein Rätsel darstellt aber hier irrelevant ist\footnote{Ob unsere Physik mit dem Urknall geschaffen wurde oder bereits vorher bestand ändert nichts an der Allgegenwärtigkeit derselben (siehe auch \textit{Paralleluniversum}).}, über die Formung unseres Planeten, bis hin zur Evolution.

Die Existenz eines Geistes kann ebenfalls ausgeschlossen werden, sein Begriff existiert aus demselben Grund wie griechische Götter. Denn obwohl biologische Prozesse wie Zellteilung oder unser Nervensystem aufgrund der Komplexität noch nicht bis ins Detail erforscht sind, spricht alles dafür, dass sogar unser (menschliches) Verhalten, mitsamt unseren bewussten und unbewussten Denkprozessen inkl. der Selbstreflexionen, in unserem physikalischen Modell erklärbar ist und beispielsweise durch eine Turing"=Maschine reproduziert werden könnte. Beziehungsweise präziser ausgedrückt: Es spricht bisher absolut nichts dagegen, denn alle jemals beobachteten oder aus der Vergangenheit rekonstruierten Ereignisse passen zusammen, ohne dass eine unbekannte Macht einmal hätte eingreifen müssen.

Da jedoch eine gewisse Unsicherheit vorliegt, vor Allem bei der Modellierung von Quanteneffekten und biologischen Prozessen, dürfen wir uns nicht anmaßen, dort grob fahrlässig eingreifen zu dürfen.

\subsubsection{Zielbestimmung}

Nachdem man einsehen muss, dass es keinen Gott und keine andere höhere Macht gibt, welche mit uns ein Ziel verfolgen könnte, und auch die Existenz eines Geistes nicht erforderlich ist, um unser Dasein und Handeln zu erklären, bleibt uns nicht mehr viel übrig, an das wir uns festhalten können. Aus Sicht des erkannten und angenommenen Weltbildes können wir machen, was wir wollen. Aufgrund der erwähnten Unsicherheit müssen wir aber vorsichtig sein, also gehen wir von der grundlegensten sicheren Beobachtung aus, die wir machen können.

Durch das Studium der Entwicklung aller lebenden Wesen können wir relativ sicher eine Gemeinsamkeit feststellen, welche man zum Ziel bzw. Recht erheben kann: Die Fortpflanzung seiner selbst bzw. den Erhalt seiner Spezies. Welches von beiden das jeweils andere impliziert, spielt dabei keine Rolle. Mit dieser Grundlage kann man alle sozialen Strukturen erklären (nicht nur menschliche). Der Umgang mit anderen Wesen und der Umwelt, insbesondere seiner Nachkommen, erfolgt hierbei zweckdienlich.

Dieser einprogrammierte Zweck ist natürlich wieder nur eine logische Konsequenz der Selektion, aber das einzige, woran wir uns festhalten können. Aufgrund der erwähnten Unsischerheiten sollten wir aber keine der biologischen Entwicklung entgegengesetzten Entscheidungen treffen, also nehmen wir diese Populationserhaltung und "~regelung als das allgemeine Ziel von Lebewesen an. Für die Konservativen und Gläubigen unter uns ist diese Ansicht ebenfalls keinesfalls schädlich, da die meisten Glaubenssätze und Moralvorstellungen übergeordnete Notwendigkeiten darstellen, um diesem konstruierten Zweck gerecht zu werden.

Auf dieser Basis können wir nun also bessere (weniger eingeschränkte) Gesetze ausarbeiten, als sie uns durch unser instinktives Verhalten mitgegeben wurden (eine Argumentation zum Sinn von Gesetzen wurde bereits in \vref{sec:situation/laws} gegeben). Zwecks Robustheit muss das Ganze möglichst allgemeingültig und systematisch formuliert werden.

Aufgrund der prinzipiell erlaubten Willkür, können wir unter Einhaltung erwähnter Grenzen allen Mitgliedern, die sich an die festgelegten Regeln halten, von aktiver Schädigung befreien und allgemein gegen die Herstellung maximaler Zufriedenheit streben. Je nach genetischer Verwandtheit werden die Rechte auf andere Lebensformen entsprechend übertragen, d.\,h. insbesondere, dass auch Tiere ein Recht auf Wohlstand haben. Ob es erlaubt sein wird, sie zu töten um sie zu verspeisen, hängt von vielen Wichtungsfaktoren ab, welche, wie alles andere auch, mit Hilfe des kompetenzgewichtet"=demokratischen Wahlsystemes festgelegt werden. Dabei werden von den Mitgliedern möglichst elementare Entscheidungen getroffen und nach oben zu allgemeineren Schlussfolgerungen synthetisiert. Dies wird in \vref{chap:architecture} näher erläutert.

\subsection{Implikationen}\label{sec:basis/implications}

Nur lebende biologische Organismen werden als Mitglieder akzeptiert, sofern sie dem zustimmen. Das Erbgut vorhandener Mitglieder definiert indirekt den Schutzgrad aller anderen Wesen.

Das System wird mit einer beliebigen Mitgliederzahl arbeiten, kann zunächst also beispielsweise auf kommunaler oder nationaler Ebene angewendet werden. Es wird im Wohle seiner Mitglieder arbeiten und kann jederzeit auf weitere Wesen ausgedehnt werden.

Darüberhinaus ist es wichtig zu berücksichtigen, dass der Sinn unserer Existenz dabei mit Religion und Natürlichkeit aufrecht erhalten werden muss, wie die Folgenden Abschnitte darlegen.

\subsubsection{Religion}

Religion (nicht unbedingt im klassischen Sinne) ist wichtig für hochintelligente Wesen, auch wenn es keine Götter gibt, um mit der Tatsache, dass eigentlich alles egal ist, umgehen zu können. Sie füllt sozusagen die Wissenslücke zur Existenzbegründung des Universums.

\subsubsection{Genetik}

Eingriffe in die Genetik des Menschen lassen uns vom Evolutionsgedanken entgleisen, auf dem das System aber fundiert ist.

\section{Regelwerk}\label{sec:rulebook}

Das Regelwerk des Specios formuliert die Grundgesetze und legt den Umgang mit anderen Systemen und Lebewesen fest. Aufbauend darauf können dann spezielle auf die jeweilige Umgebung angepasste sogenannte \myindexednote{Supera Specios}Supera Specios erstellt werden, die eine realistische politische und gesetzliche Umgebung schaffen, sich aber vollständig, also kompromisslos, an das Regelwerk halten müssen. Die Schaffung eines Supera Specios wird in \vref{chap:architecture} beschrieben.

Das nachfolgende Regelwerk stammt aus einer alten Konzeptbeschreibung, veranschaulicht aber gut die Idee hinter Specios.

\begin{enumerate}
  \item Der Wohlstand exisitierender Lebewesen muss so gut wie möglich gewahrt werden, sodass eine relative Gleichberechtigung zwischen allen Wesen entsteht.
  \begin{enumerate}
    \item Dabei müssen Mitglieder eines Supera Specios, im folgenden Mitglieder genannt, dafür sorgen, dass alle Mitglieder erhalten bleiben, d.h. es dürfen auch keine Opferungen gegen den Willen des Opfernden stattfinden.\label{enum:rulebook/survive}
    \begin{enumerate}
      \item Unter "`erhalten bleiben"' versteht sich die Erhaltung eines minimalen Wohlstandslevels, welcher über folgenden beispielhaften Zuständen liegt: Tod, irreversible Verletzungen (körperliche wie seelische).
    \end{enumerate}
    \item Der Grad des zu wahrenden Wohlstandes hängt von der Wichtigkeit des jeweiligen Lebewesens ab.
    \begin{enumerate}
      \item Die Wichtigkeit eines Lebewesens berechnet sich wie folgt: Entwicklungsgrad * Intelligenz.
      \item Die Wichtigkeit einer Gattung berechnet sich wie folgt: Durchschnittliche Wichtigkeit ihrer Lebewesen / Prozentualer Anteil dieser Gattung in ihrem Lebensraum.
      \item Die Wichtigkeit von Mitgliedern ist etwas höher als die Wichtigkeit gleichwertiger Nicht"=Mitglieder.
    \end{enumerate}
  \end{enumerate}
  \item Ein Supera Specios kann auf weitere Wesen erweitert werden, sofern beide Parteien einverstanden sind und jedes Mitglied das Specios akzeptiert.
  \begin{enumerate}
    \item Aus \vref{enum:rulebook/survive} folgt, dass ein Supera Specios nicht (wieder) aufgeteilt werden kann, eine Verschmelzung muss also sorgfältig durchdacht werden.
    \item Direkte Verwandte eines jeden Mitgliedes oder Arten mit biologisch ähnlichem Aufbau bekommen entsprechend des Verwandtschaftsgrades (biologischer Übereinstimmung) einen entsprechenden Schutzstatus (welcher aufgehoben werden kann, sofern entsprechendes Wesen die Absicht hat gegen \vref{enum:rulebook/survive} bzgl. eines der Mitglieder zu verstoßen) und die Empfehlung in das System integriert zu werden.
    \begin{enumerate}
      \item Neugeborene eines Mitgliedes werden automatisch in das Supera Specios integriert.
    \end{enumerate}
  \end{enumerate}
\end{enumerate}
