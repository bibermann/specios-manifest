\chapter{Konzept}\label{chap:concept}

Dieses Kapital wird eine Lösung skizzieren, welche die in \vref{sec:consequences} dargelegten Anforderungen erfüllt.

Ich nennen das System \textit{Specios, la specio sistemo}\footnote{"`Specios, la specio sistemo"' bedeutet "`Specios, das Arten-System"', wobei \textit{Art} im Sinne von \textit{Spezies} zu verstehen ist, und \textit{System} u.\,A. im politischen Sinne. Also ein Verwaltungssystem für jede Art von Lebewesen. Wenn man möchte, kann man \textit{Specios} selbst auch ähnlich eines rekursiven Akronyms betrachten, es steht für "`\textbf{Specio} \textbf{s}ystemo por \textbf{s}inkonservo, \textbf{p}rospero, \textbf{e}galrajteco kaj la \textbf{c}elado al \textbf{i}dentigado de \textbf{o}bskureco"' (epo.\index{Esperanto} Arten-System zur Selbsterhaltung, Wohlstand, Gleichberechtigung und dem Bestreben nach der Identifizierung der Unbekanntheit). Zu beachten ist, dass "`specios"' nicht esperanto ist, bezeichnet also auch nicht die Mehrzahl von "`specio"', dies wäre "`specioj"'.}.

\section{Grundlage}\label{sec:basis}

\subsection{Ziel}\label{sec:basis/aim}

Nachdem man einsehen muss, dass es keinen Gott und keine andere höhere Macht gibt, welche mit uns ein Ziel verfolgen könnte, und auch die Existenz eines Geistes nicht erforderlich ist, um unser Dasein und Handeln zu erklären, bleibt uns nicht mehr viel. Da nicht nur unser Gehirn, sondern selbst eine einzelne lebende Zelle physikalisch extrem schwer erfassbar ist und möglicherweise diverse Quanteneffekte eine Rolle spielen, die wir noch garnicht vollständig, zumindest in der Gesamtheit unseres Organismus, verstehen, dürfen wir uns aber nicht anmaßen, dort grob fahrlässig eingreifen zu dürfen.

Unter Beobachtung der Entwicklung aller lebenden Wesen können wir relativ sicher eine Gemeinsamkeit feststellen, welche man zum Ziel bzw. Recht erheben kann: Die Fortpflanzung seiner selbst bzw. den Erhalt seiner Spezies. Welches von beiden das jeweils andere impliziert, spielt dabei keine Rolle. Mit dieser Grundlage kann man alle sozialen Strukturen erklären (nicht nur menschliche) und natürlich auch eine bessere ausarbeiten, welche Zwecks Robustheit möglichst allgemeingültig und systematisch zu formulieren ist.

Aufgrund der prinzipiell erlaubten Willkür, können wir unter Einhaltung erwähnter Grenzen allen Mitgliedern, die sich an die festgelegten Regeln halten, von aktiver Schädigung befreien und gegen die Herstellung maximaler Zufriedenheit streben. Je nach genetischer Verwandtheit werden die Rechte auf andere Lebensformen entsprechend übertragen, d.h. insbesondere, dass auch Tiere ein Recht auf Wohlstand haben. Ob es erlaubt sein wird, sie zu töten um sie zu verspeisen, hängt von vielen Wichtungsfaktoren ab, welche, wie alles andere auch, mit Hilfe des kompetenzgewichtet-demokratischen Wahlsystemes festgelegt werden. Dabei werden von den Mitgliedern möglichst elementare Entscheidungen getroffen und nach oben zu allgemeineren Schlussfolgerungen synthetisiert.

\section{Regelwerk}\label{sec:rulebook}
