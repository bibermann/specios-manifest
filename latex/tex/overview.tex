\chapter{Übersicht}\label{chap:overview}

Dieses Kapitel bietet eine Übersicht über das System.

\section{Einleitung}

Das Specios ist ein universelles Arten-System, d.\,h. es definiert Verhaltensregeln und legt Handlungsnotwendigkeiten fest, die erforderlich sind, um Wohlstand und Gleichberechtigung für jedes seiner Mitglieder sicherzustellen. Der Kern ist unabhängig von der Art des Lebewesens oder dessen Aufenthaltsort, denn wie könnte ein System jemals als "perfekt" angesehen werden, wenn es ausschließlich für den Menschen, die Erde bzw. unsere derzeitige Sicht auf die Gesellschaft entworfen würde?

\begin{quote}
"`Es ist kein Zeichen von Gesundheit, an eine zutiefst kranke Gesellschaft gut angepasst zu sein."'\\
--- Jiddu Krishnamurti
\end{quote}

Specios basiert auf aktuellen wissenschaftlichen Erkenntnissen über Gott und die Welt und bietet ein robustes Fundament. Es wird computergestützt arbeiten, bietet jedoch Notpläne, die einem Missbrauch entgegenwirken. Die Daten, die Specios benötigt, werden demokratisch von allen Mitgliedern bestimmt, wobei Kompetenzen berücksichtigt werden. Mit Hilfe des Internets ist es jederzeit offen für das ganze Volk - genau das ist der Trumpf und unsere einzige Gelegenheit aus den Fängen des Kapitalismus zu entweichen. Es ist möglich, durch vorübergehenden Verzicht auf übertriebenen Luxus, kurzfristig ein Mindest-Wohlstandslevel zu erreichen.

\begin{quote}
"`Die Welt hat genug für jedermanns Bedürfnisse, aber nicht für jedermanns Gier."'\\
--- Mahatma Gandhi
\end{quote}

Langfristig muss man weg kommen von der Profitmaximierung, man muss das tun, was wichtig ist für die Ziele des Specios, und damit im Endeffekt das Beste für jedes einzelne Mitglied. Es dürfen keine egoistischen Machtstellungen möglich sein. Das Venus-Projekt zeigt beispielsweise auf, dass wir technisch in der Lage sind, viel mehr Wohlstand zu erzeugen - mehr als ihn die wohlhabendsten Menschen heute haben. Unser aktuelles politisches System, zusammen mit dem Geldsystem, hindert uns jedoch gewaltig, da viele Zielsetzungen per Definition widersprüchlich sind. Durch eine repräsentative Darstellung der von Specios berechneten Notwendigkeiten, basierend auf grundlegenden Daten die von den Mitgliedern geliefert werden, muss sich die aktuelle Politik früher oder später danach richten.

\section{Weitere Tabellen}



%\verb|++|
%(siehe \vref{sec:rollout_mathematics}).

%\begin{figure}
%\caption[Überblick/Datenbank]{Überblick über die vom Programm verwendeten Tabellen}
%\includegraphics[width=\textwidth]{gfx/overview_database.png}
%\label{fig:overview_database}
%\end{figure}