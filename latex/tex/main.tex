% Anmerkungen:
%   Empfohlende Pakete:
%     cm-super für sauberere Schriftdarstellung

\documentclass[fontsize=10pt,paper=a4,DIV=12,abstract=true,toc=bibliography,toc=listof,toc=index]{scrreprt}

% Fonts, Sprachunterstuetzung
\usepackage[utf8]{inputenc} %Deutsche Umlaute im Quellext
\usepackage[T1]{fontenc}
%\usepackage{lmodern}
\usepackage[ngerman]{babel} %Deutsche Silbentrennung, etc.
\usepackage{babelbib} %Mehrsprachige Literaturliste

% Graphiken und Farbe
\usepackage{graphicx} %Graphiken
\usepackage{color} %Farbe
\usepackage{pdfpages} %PDF-Seiten einbinden

% Mathematisches
\usepackage{amsmath}
\usepackage{amssymb} % mathfrak + amsfonts + spezielle Symbole
\usepackage{amsxtra} % Weitere Extrasymbole
\usepackage{mathrsfs} % \mathscr - normal ist \mathcal http://en.wikipedia.org/wiki/File:Mathscr-vs-mathcal.png
\usepackage{amsthm} % Definiert sind
%theorem, corollary, definition, definitions, fact, example,
%examples, Problem, Loesung, Definition, Satz, Beweis,
%Folgerung, Lemma, Fakt, Beispiel, and Beispiele.
%Beispiel: \proof{Ich habe etwas bewiesen.\qed}

% Mathematische Schrifterweiterungen
\usepackage{dsfont} %\mathds{A} alternativ zu mathbb
\usepackage{bm} %\bm{A} Boldface im Mathemodus

% Verbesserte Enumerate-Umgebung
\usepackage{enumerate}

% Verbesserte Tabellen- und Equation-Umgebung
\usepackage{array}
\usepackage{eqnarray}
\usepackage{longtable}

% Diagramme
\usepackage[all]{xy}

\usepackage{url}
\usepackage{makeidx}
\usepackage{fancyhdr}

\definecolor{darkgreen}{rgb}{0,0.7,0} 
\definecolor{darkred}{rgb}{0.7,0,0} 
\definecolor{darkblue}{rgb}{0,0,0.7} 
\definecolor{lightgrey}{rgb}{0.97,0.97,0.97} 

\usepackage[ngerman]{varioref}
%\usepackage[pdfborder=000, pdftex=true, plainpages=false, colorlinks=true, linkcolor=black, filecolor=black, urlcolor=black, citecolor=black, backref]{hyperref}
\usepackage[colorlinks=true,linkcolor=blue,citecolor=red,urlcolor=darkgreen]{hyperref}
\usepackage[ngerman]{cleveref}


\makeindex

\addtokomafont{caption}{\raggedright} % Bildunterschriften linksbündig

\newcommand{\todo}[1]{\textcolor{red}{TODO: #1}}
\newcommand{\degree}{\ensuremath{^\circ}}

%\renewcommand{\figurename}{Abbildung}
%\renewcommand{\tablename}{Tabelle}
%\renewcommand\contentsname{Inhaltsverzeichnis}



%\titlehead{fineshift M-Bus Reader v3.1.1 Dokumentation v2.0.0}
%\subject{Dokumentation\thanks{fineshift SLP-Rollout v1.4.1 Dokumentation v2.0.0}}
\subject{Manifest}
\title{Specios}
\author{Fabian Sandoval Saldias}
\date{17. Februar 2011}
%\publishers{fineshift \vtop{\vskip-25pt\hbox{\includegraphics[height=40pt]{gfx/fineshift.png}}}}
%\publishers{\textrm{}\vtop{\vskip-25pt\hbox{\includegraphics[height=40pt]{gfx/fineshift.png}}\vskip25pt} fineshift}
%\publishers{\textrm{}\vtop{\vskip-82.5pt\hbox{\includegraphics[height=160pt]{gfx/fineshift.png}}\vskip82.5pt}\textrm{}}
\publishers{\textrm{}\vtop{\vskip-43pt\hbox{\includegraphics[height=80pt]{gfx/fineshift.png}}\vskip44pt}\textrm{}}
%\dedication{Vielen Dank}



%\renewcommand{\subsectionmark}[1]{
%  \markright{\thesubsection{} #1}
%}

%\setcounter{secnumdepth}{4}
%\setcounter{tocdepth}{4}


\pagestyle{fancy} %eigener Seitenstil
\fancyhf{} %alle Kopf- und Fußzeilenfelder bereinigen
\fancyhead[L]{\textsc{\nouppercase{\leftmark}}} %Kopfzeile links
\fancyhead[R]{\textsc{\nouppercase{\rightmark}}} %Kopfzeile rechts
\fancyfoot[C]{\thepage} %Seitennummer
\renewcommand{\footrulewidth}{0.4pt} %untere Trennlinie

\fancypagestyle{plain}{
  \fancyhf{} %alle Kopf- und Fußzeilenfelder bereinigen
  \fancyfoot[C]{\thepage} %Seitennummer
  \renewcommand{\headrulewidth}{0pt} %obere Trennlinie
  \renewcommand{\footrulewidth}{0.4pt} %untere Trennlinie
}


\begin{document}



\pdfbookmark{Startseite}{title}\maketitle


\pdfbookmark{\abstractname}{abstract}
\begin{abstract}
Unser Gesellschaftsmodell hat ausgedient. Wir werden von einer Minderheit beherrscht, welche natürlich alles tut, um ihre Machtposition zu behalten. Dies ist ein ganz natürliches Verhalten und wird durch unser profitorientiertes System sogar unterstützt. Vor der Industrialisierung war das sogar sinnvoll. Heute aber führt dies zu Massenversklavung, Wohlstandsverlust und riesiger Ressourcenvernichtung.

Dieses Manuskript wird diesen Wandel erklären, warum sich alternative Systeme wie der Kommunismus nicht etablieren konnten und warum eine Veränderung zwingend notwendig ist. Schließlich wird eine mögliche Lösung präsentiert: das Specios. Mit ihm wird ein grundlegend neues Gesellschaftskonzept vorgelegt und erläutert, wie es in wenigen Jahren alle derzeitigen politischen und Rechtssysteme erfolgreich unterwandern und ein nachhaltig wohlständiges Leben auf dem gesamten Planeten und darüber hinaus garantieren kann.

\end{abstract}


\pdfbookmark{\contentsname}{toc}\tableofcontents
%\clearpage


%\begin{enumerate}
% \item Erster Block. Hier wird die Marke gesetzt.\label{verweis}
% \item Mittels \pageref{verweis} kann auf die Seitennummer des fraglichen Labels verwiesen werden.
% \end{enumerate}


% \vspace*{10cm}
 % \begin{longtable}{|l|r|c|p{2cm}|}
 % \hline
 % Linksbündige Spalte.&Rechtsbündige Spalte&Zentrierte Spalte&Parbox\\
 % \hline
 % Kurzer Text.&Kurzer Text.&Kurzer Text.&Kurzer Text.\\
 % \hline
 % Text.&Text.&Text.&In diesem Felde nun ein elendslanger text, um Umbrüche innerhalb eines Feldes zu erzeugen.\\
 % \hline
 % Text.&Text.&Text.&der Befehl $\backslash$vspace*\{\} weiter oben im Quellcode hat einem Umbruch vor dieser Zeile bewirkt.\\
 % \hline
 % \end{longtable}


% \begin{enumerate}
% \item Zweiter Block. Hier wird nach \vref{verweis} verwiesen.
% \item Mittels \pageref{verweis} kann auf die Seitennummer des fraglichen Labels verwiesen werden.
% \end{enumerate}


% \section{Erster Abschnitt}\label{sec:firstsection}
% Wenn $a=2$ und $b=3$ dann gilt:
% \begin{equation}\label{eq:simpleeq}
% a+b =2+3=5
% \end{equation}
% Wie in \cref{eq:simpleeq} gezeigt \ldots
% \newpage
% \section{Zweiter Abschnitt}\label{zweiter Abschnitt}
% \Cref{eq:simpleeq}\vpageref{eq:simpleeq} verdeutlichte bereits \ldots


% So sind 96 \% der bundesdeutschen Schulen mit Computern ausgestattet. \cite[S. 7]{bmbf} ...


\chapter{Übersicht}\label{chap:overview}

Dieses Kapitel bietet eine Übersicht über das System.

\section{Haupttabellen}

Die wichtigste Tabelle ist die "`Kunden"'"=Tabelle, in der alle Kunden mit allen nötigen Informationen, z.B. dem anzuwendenen Standardlastprofil, gelistet sind. Slp"=Rollout unterstützt Sie dabei, die Informationen über Kunden zu vervollständigen.

Die Tabelle "`Gesamtmengen"' enthält den gemessenen Gesamtverbrauch eines Kunden für jedes Jahr. Mit Slp"=Rollout können die Daten importiert werden.
Der Verbrauch der vergangenen 3 Jahre wird aus dieser Tabelle verwendet, um den Kundenwert zu berechnen. Der Kundenwert ist ein Maß für die Größe des Verbrauches im Verhältnis zu den mit Hilfe des Kundenmodells mathematisch berechneten Referenz"=Werten.

Die Tabelle "`Stundenwerte"' enthält die ausgerollten Verbrauchswerte eines Kunden für jede Stunde eines Jahres, mit einer Genauigkeit von 24h.

\section{Weitere Tabellen}

Der wichtigste Faktor bei der Schätzung des Verbrauchs ist die durschnittliche Tagestemperatur. Sie wird mit Hilfe von Wetterstationen erfasst, dessen Daten in der Tabelle "`Tagestemperaturen"' gepflegt werden müssen.

Jeder Eintrag in dieser Tabelle ist mit einer "`Temperaturquelle"' verknüpft, welche in der gleichnamigigen Tabelle hinterlegt wird.


Die Tabellen "`PLZBundelandZuordnungen"' und "`OrtBundelandZuordnungen"' dienen Slp"=Rollout zur automatisierten Zuordnung von Kunden zu den Bundesländern anhand der Postleitzahl oder des Ortes. Die Tabelle "`VertragsnameSLPZuordnungen"' enthält entsprechend Zuordnungen von Vertragsname"=Bestandteilen zu dem zu verwendenden SLP.

Die Tabelle "`Feiertage"' enthält alle festen (optional bundeslandspezifischen) Feiertage. Bewegliche christliche Feiertage berechnet das Programm selbstständig.

Alle anderen Tabellen beinhalten die von der TU München ermittelten Parameter für die SLP"=Ausrollung bzw. die Namen, die sich hinter Bundesland- und SLP"=Kennungen verbergen.

\section{Parameter, Auswirkungen}

Der ausgerollte Verbrauchswert eines Tages ist abhängig vom Bundesland, vom gewählten SLP und dessen Ausprägung (\verb|++|, \verb|+|, \verb|o|, \verb|-| oder \verb|--|), vom Wochentag/Feiertag, von der Durchschnittstemperatur und vom Kundenwert (siehe \vref{sec:rollout_mathematics}).

Das bedeutet, dass, wenn für einen ganzen Monat die selbe Temperatur angenommen werden würde, die einzige lokal greifende Variable der Wochentag wäre. Sofern keine Feiertage dazwischen liegen, würde sich das Verbrauchsschema also alle 7 Tage wiederholen, beim Ein- und Mehrfamilienhaushalt würde der Verbrauch sogar an jedem Tag des Monats identisch sein, da sie wochentagsunabhängig sind.

Kunden mit denselben Parametern (Bundesland, Temperaturquelle, SLP, SLP"=Ausprägung) haben relativ gesehen exakt dasselbe Profil, lediglich die Absolutwerte sind je nach Kundenwert unterschiedlich (aber gleichmäßig) skaliert.

%\begin{figure}
%\caption[Überblick/Datenbank]{Überblick über die vom Programm verwendeten Tabellen}
%\includegraphics[width=\textwidth]{gfx/overview_database.png}
%\label{fig:overview_database}
%\end{figure}
\chapter{Vorgehensweisen}\label{chap:strategies}

\section{Allgemeine Daten verstehen und pflegen}

\subsection{Feiertage}\label{sec:holidays}

Feiertage entsprechen bei der Berechnung einem Sonntag, d.h. Feiertage an Sonntagen müssen nicht berücksichtigt werden.
Folgende beweglichen christlichen bundesweiten Feiertage werden vom Programm automatisch berechnet: Karfreitag, Ostermontag, Christi Himmelfahrt, Pfingstmontag.

Alle anderen Feiertage müssen in die Tabelle "`Feiertage"' eingetragen werden. Dabei kann das Bundesland=null gesetzt werden, sofern der jeweilige Feiertag bundesweit gilt, und/oder das Jahr=null gesetzt werden, sofern der jeweilige Feiertag jedes Jahr am selben Datum stattfindet.

\subsection{Temperaturen}\label{sec:temperatures}

Zur Berechnung des Kundenwertes müssen explizite Tagestemperaturen von mindestens einem Jahr innerhalb des Zeitraumes von 3 Jahren vor dem Ausrolljahr bis zum Ausrolljahr selbst vorliegen, das ist eine Zeitspanne von 4 Jahren, bzw. praktisch eher 3 Jahre, da für das Ausrolljahr selbst normalerweise noch keine gemessenen Temperaturen zusammen mit dem Jahresverbrauch vorliegen. Nur Jahre, in denen beides, explizite Temperaturangaben und der gemessene Jahresverbrauch, vorhanden sind, werden für die Kundenwertberechnung verwendet. Als explizite Temperatur gelten alle Temperaturen mit einer Jahresangabe. Temperaturangaben mit Zuordnung zu der Messstation, die dem Kunden zugeordnet ist, werden gegenüber messstationsunabhängigen Temperaturangaben bevorzugt.

Optimalerweise sollten Temperaturwerte vom 29.---31.12. des 4. Jahres vor dem Ausrolljahr ebenfalls vorliegen, da alle Temperaturwerte von Slp"=Rollout nicht direkt, sondern mit Hilfe einer geometrischen Reihe über den aktuellen und die vergangenen 3 Tage ermittelt wird (wobei der letzte dieser Tage doppelt so hoch gewichtet wird wie der vorletzte usf.), um die Wärmespeicherfähigkeit von Gebäuden zu berücksichtigen. Der Algorithmus ist dabei tolerant gegenüber fehlenden Temperaturangaben, so dass für bis zu 3 aufeinanderfolgende Tage Temperaturwerte fehlen können, ohne dass Slp"=Rollout versagt.


Wie gesagt können die Tempetraturwerte in die Tabelle "`Tagestemperaturen"' also ohne die Angabe der Temperaturquelle und/oder des Jahres eingetragen werden.

Geschätzte Temperaturangaben (z.B. des laufenden Jahres) sollten allerspätestens dann, nachdem eine Gesamtmenge für dieses Jahr eingetragen wurde, auf gemessene Werte aktualisiert werden.

Neue Temperaturquellen müssen in der Tabelle "`Temperaturquellen"' eingetragen werden --- wichtig ist das Feld Adresse für die manuelle Zuordnung und die geographische Position für die automatische Messstationszuordnung: Latitude (geographische Breite (N---S), in Grad), Longitude (geographische Länge (W---E), in Grad) und Altitude (Höhe über Normalhöhennull, in Metern).

\section{Kunden hinzufügen und Kundendaten ändern}\label{sec:modifyclients}

Alternativ zu den im Folgenden genannten Vorgehensweisen, kann die Kunden-Tabelle auch manuell bearbeitet werden.

Andernfalls muss das Programm Slp"=Rollout gestartet werden.

In den Einstellungen (erscheinen beim Programmstart) muss der Pfad zur Access"=Hauptdatenbank angegeben werden.

Das Kontrollkästchen "`Neue Daten"' muss markiert werden, wenn neue Daten (Kunden oder Gesamtmengen) hinzugefügt werden sollen.

Im Gruppenrahmen dieses Kontrollkästchens kann dann der Tabellenname der neuen Kundendaten und das Jahr, für welches die Gesamtmengen angegeben sind, eingestellt werden.

Anschließend auf die Schaltfläche "`Übernehmen \verb|>>|"' klicken. Es wird eine Verbindung zur Datenbank aufgebaut und einige Tabellen mit Parametern, SLP"=Namen usw. werden geladen. Außerdem werden diverse Abfragen ausgeführt, um die Kunden"=Tabellen u.ä. im Programm zu füllen.

\subsection{Kunden hinzufügen}\label{sec:modifyclients/add}

In der Registerkarte "`Kunden vervollständigen"' werden alle unvollständigen und neuen Kunden angezeigt, wo sie direkt bearbeitet werden können. Mehrere Kunden können dazu gleichzeitig ausgewählt werden.

Angepasst werden müssen "`Bundesland"', "`SLP"', "`SLP"=Ausprägung"' und "`Temperaturquelle"'. Einige Felder werden automatisch versucht auszufüllen, siehe dazu \vref{sec:clientautocompletion}.

Unterhalb der Tabelle werden für die ausgewählten Kunden all diese Felder für die aktuelle Auswahl angezeigt. Gibt es Abweichungen in der Auswahl, bleibt das entsprechende Feld leer, ansonsten wird der gemeinsame Wert angezeigt.

Über Dropdown"=Listen können die Werte nun geändert werden. Mit einem Klick auf "`Übernehmen \verb|>>|"' werden sie auf folgende Weise übernommen: Ist ein Feld leer, wird dieses Feld nicht angefasst, enthält ein Feld einen Wert, werden alle markierten Kunden mit diesem neuen Wert überschrieben.

Hinweis: Über die Buttons "`nicht ändern"' kann ein gesetztes Feld wieder geleert werden, um es nicht anzufassen.


Die Schaltfläche "`Datenbank aktualisieren"' überträgt nun (alle) Kunden in die Kundentabelle, wobei ihnen eine eindeutige ID zugewiesen wird, falls der Kunde noch nicht exisitert; andernfalls wird er aktualisiert. Ist das Kontrollkästchen "`Auch unausgefüllte Kunden übertragen"' nicht markiert, werden Kunden, bei denen Bundesland, SLP und Temperaturquelle nicht gesetzt ist und SLP"=Ausprägung gleich "`\verb|o|"' ist, nicht übertragen. Andernfalls werden alle Kunden übertragen.

\subsection{Kunden ändern}\label{sec:modifyclients/modify}

Die Eigenschaften, oben Felder genannt, aller übertragenen Kunden können jederzeit geändert werden. Die Ausroll"=Funktionalität ist Abhängig von diesen Eigenschaften ("`Bundesland"', "`SLP"', "`SLP"=Ausprägung"' und "`Temperaturquelle"').

Die Vorgehensweise ist dieselbe wie in \vref{sec:modifyclients/add} beschrieben.

\subsection{Gesamtmengen hinzufügen bzw. ändern}

In der Registerkarte "`Gesamtmengen übertragen"' werden die Gesamtmengen für alle Kunden angezeigt, die sich in der Datenbank befinden und für die es in der neuen Tabelle (siehe \vref{sec:modifyclients}) abweichende oder neue Gesamtmengen für das jeweilige Jahr gibt.

Die Schaltfläche "`Auswahl in Datenbank übertragen"' überträgt alle ausgewählten Gesamtmengen in die Datenbank, dabei werden sie ggf. überschrieben, falls für das jeweilige Jahr bereits Daten vorhanden waren.

Hinweis: Mit der Schaltfläche "`Alle auswählen"' können alle Gesamtmengen schnell ausgewählt werden.

\section{SLPs erraten}

Hierfür muss das Programm Slp"=Rollout gestartet werden.

Hinweis: Ggf. muss vorher der Tabellenname in der Datei "`comparison\_query.sql"' angepasst werden.

Die Einstellungen im Gruppenrahmen "`Ausrollen"' müssen angepasst werden.

Anschließend auf die Schaltfläche "`Übernehmen \verb|>>|"' klicken. Einige Daten werden geladen, siehe \vref{sec:modifyclients}.


In der Registerkarte "`SLPs erraten"' werden nun alle Kunden aufgelistet, für die vollständige ausroll"=spezifische Informationen vorliegen (bis auf SLP, welches null sein muss) und für die es mindestens eine Jahres"=Gesamtmenge gibt.

Hier können nun alle Kunden markiert werden, für die das best"=passendste SLP berechnet werden soll.

Hinweis: Mit der Schaltfläche "`Alle auswählen"' können alle Kunden schnell ausgewählt werden.

Bei Klick auf die Schaltfläche "`Bis zu 70 Kombinationen ausrollen und bestes SLP setzen"' wird eben dies getan. Im Ordner "`comparison\_results"' können die Ergebnisse manuell eingesehen werden.

\section{SLPs ausrollen}

Hierfür muss das Programm Slp"=Rollout gestartet werden.

In den Einstellungen (erscheinen beim Programmstart) muss der Pfad zur Access"=Hauptdatenbank angegeben werden.

Das Kontrollkästchen "`Neue Daten"' muss nicht markiert werden.

Anschließend auf die Schaltfläche "`Übernehmen \verb|>>|"' klicken. Einige Daten werden geladen, siehe \vref{sec:modifyclients}.


In der Registerkarte "`SLPs ausrollen"' werden nun alle Kunden aufgelistet, für die vollständige ausroll"=spezifische Informationen vorliegen und für die es mindestens eine Jahres"=Gesamtmenge gibt.

Hier können nun alle Kunden markiert werden, die ausgerollt werden sollen.

Hinweis: Mit der Schaltfläche "`Alle auswählen"' können alle Kunden schnell ausgewählt werden.

Ist das Kontrollkästchen "`Jeden Kunden vorher bestätigen"' markiert, wird vor dem Ausrollen eines jeden Kunden eine Zusammenfassung der ausgerollten Daten angezeigt, die bestätigt werden muss. Die Zusammenfassung enthält z.B. die Information welche Jahre für die Kundenwertberechnung verwendet wurden (siehe \vref{sec:rollout_mathematics}). Wenn sehr viele Kunden ausgerollt werden sollen, empfiehlt es sich dieses Kontrollkästchen nicht zu markieren.
%Ist das Kontrollkästchen "`Nur Tageswerte"' markiert, werden nur Tageswerte geschrieben (das Feld "`Stunde"' wird auf 0 gesetzt und das Feld "`Jahresstunde"' beginnt beim 1. Tag des Jahres bei 12).

Bei Klick auf die Schaltfläche "`Ausrollen"' werden nun alle ausgewählten Kunden in die Datenbank ausgerollt.
%Sollen die Kunden nicht in die Datenbank, sondern in eine CSV"=Datei ausgerollt werden, kann stattdessen die Schaltfläche "`In CSV..."' verwendet werden.

\chapter{Programm-Arbeitsweise}\label{chap:background}

\section{Automatische Kundeneigenschaften"=Zuordnung}\label{sec:clientautocompletion}

Folgende Eigenschaften können automatisiert zugeordnet werden: Bundesland, Koordinaten und SLP.

Um dies zu Bewerkstelligen, gibt es Tabellen mit regulären Ausdrücken. Das sind Vorschriften, mit denen bestimmte andere Werte des Kunden geprüft werden. Bei Erfolg wird die hinterlegte Eigenschaft gesetzt. Das ist die automatische Kundeneigenschaften"=Zuordnung.

Im speziellen gibt es die Tabelle "`PLZBundeslandZuordnungen"', mit dessen Hilfe das Programm die PLZ eines Kunden überprüft und ihm ggf. ein Bundesland zuordnet.

Ist dieser Versuch erfolglos, wird die Tabelle "`OrtBundeslandZuordnungen"' verwendet, um dasselbe anhand des Ortsnamen zu versuchen. Der Ortsname wird dafür aus dem Vertragsnamen abgeleitet.

Daneben gibt es die Tabelle "`PLZKoordinatenZuordnungen"', mit dessen Hilfe das Programm die ungefähren geographischen Koordinaten eines Kunden anhand der PLZ zuordnet.

Weiterhin gibt es die Tabelle VertragsnameSLPZuordnungen, mit dessen Hilfe das Programm anhand des Vertragsnamen ein SLP auswählen kann.

In diesen 4 Tabellen entscheidet die Spalte "`Priorität"' über die Priorität dieser Tests. Sind mehrere Tests erfolgreich, wird die Zuordnung mit der niedrigsten Prioritätszahl gewählt.

Zur Syntax der regulären Audrücke mit der verwendeten Engine siehe \url{http://qt.nokia.com/doc/4.5/qregexp.html}.

\section{Mathematik des Ausrollens}\label{sec:rollout_mathematics}

Zunächst werden alle für einen Kunden wichtigen Parameter ermittelt: Die möglichen Koeffizienten für die Sigmoid"=Funktion, die Wochentagsfaktoren und alle zutreffenden Feiertage und Tagestemperaturen für dieses und die 4 vorhergehenden Jahre. Siehe auch \vref{sec:holidays} und besonders \vref{sec:temperatures}.

Einige Werte werden entsprechend der Praxisinformation P2007/13 des BGW \cite{bgwvkup200713} gerundet.

Dann wird der Kundenwert ermittelt, welcher am Ende auf eine ganze Zahl gerundet wird. Dafür werden die Gesamtmengen der letzten 3 Jahre vor dem Ausrolljahr verwendet, sofern explizite Temperaturwerte vorliegen. Der Kundenwert ist die Summe der Gesamtmengen des betrachteten Zeitraumes dividiert durch die Summe der errechneten Tagesverbrauchszahlen, mit Hilfe der Parameter für jeden Tag eines Jahres.

Die Tagesverbrauchszahlen sind das Produkt aus dem passenden Wochenatagsfaktor (bei Feiertagen der Faktors für Sonntag) und der Sigmoidfunktion in Abhängigkeit der passenden Koeffizienten und dem arithmetischen Mittel der Tagesmitteltemperatur der letzten 4 Tage.

Die Sigmoidfunktion:
\begin{equation*}
Sigmoidfunktion = \frac{A}{(1 + \frac{B}{Temperatur-40\degree C}^C)} + D
\end{equation*}

Anschließend werden die Tagesverbrauchszahlen für jeden Tag das auszurollende Jahres ermittelt und mit dem Kundenwert multipliziert, um den Tagesverbrauch in kWh zu berechnen. Hierfür wird auf allgemeine Temperaturwerte ausgewichen, falls für das auszurollende Jahr noch keine expliziten Temperaturwerte vorliegen.

\chapter{Verschiedenes}\label{chap:remarks}

\begin{itemize}
\item Wochentagsfaktoren und Kundenwert beeinflussen das Ergebnis linear.
\item Das Bundesland beeinflusst die Feiertage und die Sigmoid"=Koeffizienten.
\item Das SLP beeinflusst die Sigmoid"=Koeffizienten und die Wochentagsfaktoren.
\item Die SLP"=Ausprägung beeinflusst die Sigmoid"=Koeffizienten.
\item Latitude und Longitude können aus einer Google"=Maps"=Adresse entnommen werden: Sie werden im Parameter "`\verb|ll|"' kodiert. Ausschnitt eines Beispieles: "`\verb|http://maps.google.de/?[...]\&ll=52.523866,13.321325&[...]|"'; hier ist Latitude = 52,523866\degree und Longitude = 13,321325\degree . Alternativ kann folgender Code in die Adresszeile des Browsers geschrieben werden, um die Koordinaten des aktuell zentrierten Punktes anzuzeigen:
\begin{verbatim}
javascript:void(prompt('',gApplication.getMap().getCenter()));
\end{verbatim}

\end{itemize}

\chapter{Changelog}\label{chap:changelog}

Im Folgenden finden Sie eine Übersicht über die Änderungen im Laufe der Versionsgeschichten der Dokumentation und des Programmes; die jüngsten Versionen befinden sich jeweils oben.

\section{Dokumentation}

\begin{itemize}

\item v2.0.0 (2010-10-12):
\begin{itemize}
\item Komplette Neuauflage
\item Für SLP-Rollout v1.4.1 (2010-06-14)
\end{itemize}

\item v1.0.0 (2009-03-24):
\begin{itemize}
\item Urwerk
\item Für SLP-Rollout v1.1.0 (2010-03-24)
\end{itemize}

\end{itemize}

\section{SLP-Rollout}

\begin{itemize}

\item v1.4.1 (2010-06-14)
\begin{itemize}
\item Ausgerollte Daten können gelöscht werden.
\item Es kann nun gewählt werden, ob Stundenwerte in eine CSV geschrieben werden sollen.
\item fixed: Kleinigkeiten, z.B. fehlerhafte Aktualisierung in der Unvollständige-Kunden-Liste
\end{itemize}

\item v1.4.0 (2010-06-09)
\begin{itemize}
\item Temperaturstation-Combobox wurde nicht angepasst bei Selektionen
\item Neue Ausroll-Option: "`Das folgendes Jahr auch für die Kundenwertberechnung benutzen"'
\item Neue Ausroll-Option: Jahresunabhängige Temperaturwerte ab bestimmtem Jahr zulassen
\item SLP-Test (bis zu 70 Kombinationen (14 SLPs * 5 Ausprägungen (aber nicht jede Kombination möglich))) und Auswahl des besten Profils (mittels Ausrollen und Vergleichsquery)
\item Jahresbereich für Kundenwertberechnung kann geändert werden
\end{itemize}

\item v1.3.1 (2010-06-02)
\begin{itemize}
\item Einstellung [bool debug\textbackslash skipDownload] entfernt und die Steuerung (Download, Import oder beides) in GUI eingebaut
\item Ausweichtemperaturwerte werden nicht mehr fest/global für ein Jahr gewählt sondern als priorisierte Liste geführt, für den Fall dass während der Berechnung längere Perioden Werte fehlen (z.B. aktuelles Jahr)
\end{itemize}

\item v1.3.0 (2010-05-20)
\begin{itemize}
\item einiges, hab aber glaubich vergessen version zu erhöhen --- lief wohl alles unter 1.2.1
\item optional zeilenweise Query-Ausführung
\end{itemize}

\item v1.2.1 (2010-04-16)
\begin{itemize}
\item (Änderungen bis hierher nicht protokolliert)
\end{itemize}

\end{itemize}

\appendix
\chapter{CSV-Dateien}\label{chap:csvfiles}\index{CSV-Format}

\section{Allgemeines}\label{sec:csvfiles/general}

Grundlegende Informationen zu den *.csv"=Dateien.

\subsection{Format}

\begin{itemize}
\item Trennzeichen: Komma
\item Escaping von Kommas: Anführungszeichen um den kompletten Wert
\item Escaping von Anführungszeichen: Doppeltes Anführungszeichen innerhalb von Anführungszeichen
\item Mehrzeilige Werte: nein
\item Zeichenkodierung: ISO 8859-1 (Latin-1)
\end{itemize}



%\listoffigures
%\listoftables
\printindex

\begin{thebibliography}{BGW/VKU, 2007}
   \bibitem[BGW, 2006]{bgwp20068}\textsc{Bundesverband der deutschen Gas- und Wasserwirtschaft} (2006): \textit{Leitfaden SLP. Praxisinformation P 2006/8 Gastransport/Betriebswirtschaft: Anwendung von Standardlastprofilen zur Belieferung nicht"=leistungsgemessener Kunden.}
   \bibitem[BGW/VKU, 2007]{bgwvkup200713}\textsc{Bundesverband der deutschen Gas- und Wasserwirtschaft; Verband kommunaler Unternehmen e. V.} (2007): \textit{Praxisinformation P2007/13 Gastransport/Betriebswirtschaft: Abwicklung von Standardlastprofilen.}
   % \bibitem[BMBF, 2003]{bmbf}"'IT-Ausstattung der allgemein bildenden und berufsbildenden Schulen in Deutschland"', http://www.schulen-ans-netz.de/neuemedien/fakten/dokus/it-ausstattung-2003.pdf, 10.03.2005
\end{thebibliography}

\end{document} 