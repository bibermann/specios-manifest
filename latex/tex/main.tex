% Anmerkungen:
%   Empfohlende Pakete:
%     cm-super für sauberere Schriftdarstellung

\documentclass[
  fontsize=10pt
  ,paper=a4
  ,DIV=12
  ,abstract=true
  ,toc=bibliography
  ,toc=listof
  ,toc=index
  %,parskip=half
  ,twoside=true
  %,open=right
]{scrreprt}

% Fonts, Sprachunterstuetzung
\usepackage[utf8]{inputenc} %Deutsche Umlaute im Quellext
\usepackage[T1]{fontenc}
\usepackage{textcomp} %z.B. Währungssymbole (Euro €, Pfund Sterling £)
%\usepackage{lmodern}
\usepackage[ngerman]{babel} %Deutsche Silbentrennung, etc.
\usepackage{babelbib} %Mehrsprachige Literaturliste

% Graphiken und Farbe
\usepackage{graphicx} %Graphiken
\usepackage{color} %Farbe
\usepackage{pdfpages} %PDF-Seiten einbinden

% Mathematisches
\usepackage{amsmath}
\usepackage{amssymb} % mathfrak + amsfonts + spezielle Symbole
\usepackage{amsxtra} % Weitere Extrasymbole
\usepackage{mathrsfs} % \mathscr - normal ist \mathcal http://en.wikipedia.org/wiki/File:Mathscr-vs-mathcal.png
\usepackage{amsthm} % Definiert sind
%theorem, corollary, definition, definitions, fact, example,
%examples, Problem, Loesung, Definition, Satz, Beweis,
%Folgerung, Lemma, Fakt, Beispiel, and Beispiele.
%Beispiel: \proof{Ich habe etwas bewiesen.\qed}

% Mathematische Schrifterweiterungen
\usepackage{dsfont} %\mathds{A} alternativ zu mathbb
\usepackage{bm} %\bm{A} Boldface im Mathemodus

% Verbesserte Enumerate-Umgebung
%\usepackage{enumerate}
\usepackage[pointlessenum]{paralist} % siehe http://kaldor.vwl.uni-hannover.de/karl/ltxmp/latex.php

% Verbesserte Tabellen- und Equation-Umgebung
\usepackage{array}
\usepackage{eqnarray}
\usepackage{longtable}

% Diagramme
\usepackage[all]{xy}

\usepackage{url}
\usepackage{makeidx}
\usepackage{fancyhdr}

\definecolor{darkgreen}{rgb}{0,0.7,0} 
\definecolor{darkred}{rgb}{0.7,0,0} 
\definecolor{darkblue}{rgb}{0,0,0.7} 
\definecolor{lightgrey}{rgb}{0.97,0.97,0.97} 

\usepackage[ngerman]{varioref}
%\usepackage[pdfborder=000, pdftex=true, plainpages=false, colorlinks=true, linkcolor=black, filecolor=black, urlcolor=black, citecolor=black, backref]{hyperref}
\usepackage[colorlinks=true,linkcolor=blue,citecolor=red,urlcolor=darkgreen]{hyperref}
\hypersetup{
	pdfpagelayout=TwoPageRight
	%,pdfstartview=Fit
}
\usepackage[ngerman]{cleveref}


%\usepackage[numbers,round]{natbib}
\usepackage[numbers,square]{natbib}
\bibliographystyle{alphadin}


\makeindex

\addtokomafont{caption}{\raggedright} % Bildunterschriften linksbündig

\newcommand{\todo}[1]{\textcolor{red}{TODO: #1}}
\newcommand{\degree}{\ensuremath{^\circ}}

%\renewcommand{\figurename}{Abbildung}
%\renewcommand{\tablename}{Tabelle}
%\renewcommand\contentsname{Inhaltsverzeichnis}


\newcommand*\myvref[1]{\nameref{#1}, \vref{#1}}



%\titlehead{fineshift M-Bus Reader v3.1.1 Dokumentation v2.0.0}
%\subject{Dokumentation\thanks{fineshift SLP-Rollout v1.4.1 Dokumentation v2.0.0}}
\subject{Manifest}
\title{Specios}
\author{Fabian Sandoval Saldias}
\date{10. März 2011}
%\publishers{fineshift \vtop{\vskip-25pt\hbox{\includegraphics[height=40pt]{gfx/fineshift.png}}}}
%\publishers{\textrm{}\vtop{\vskip-25pt\hbox{\includegraphics[height=40pt]{gfx/fineshift.png}}\vskip25pt} fineshift}
%\publishers{\textrm{}\vtop{\vskip-82.5pt\hbox{\includegraphics[height=160pt]{gfx/fineshift.png}}\vskip82.5pt}\textrm{}}
\publishers{\textrm{}\vtop{\vskip-43pt\hbox{\includegraphics[height=80pt]{gfx/fineshift.png}}\vskip44pt}\textrm{}}
%\dedication{Vielen Dank}



%\renewcommand{\subsectionmark}[1]{
%  \markright{\thesubsection{} #1}
%}

%\setcounter{secnumdepth}{4}
%\setcounter{tocdepth}{4}


\pagestyle{fancy} %eigener Seitenstil
\fancyhf{} %alle Kopf- und Fußzeilenfelder bereinigen
\fancyhead[L]{\textsc{\nouppercase{\leftmark}}} %Kopfzeile links
\fancyhead[R]{\textsc{\nouppercase{\rightmark}}} %Kopfzeile rechts
\fancyfoot[C]{\thepage} %Seitennummer
\renewcommand{\footrulewidth}{0.4pt} %untere Trennlinie

\fancypagestyle{plain}{
  \fancyhf{} %alle Kopf- und Fußzeilenfelder bereinigen
  \fancyfoot[C]{\thepage} %Seitennummer
  \renewcommand{\headrulewidth}{0pt} %obere Trennlinie
  \renewcommand{\footrulewidth}{0.4pt} %untere Trennlinie
}


% Folgende Befehle machen den Veränderten Rand durch twoside=true wieder weg.
% Wird twoside=true nicht verwendet oder soll das ganze gebunden werden, sollten
% diese Befehle entfernt werden.
\setlength{\oddsidemargin}{0cm}
\setlength{\evensidemargin}{0cm}


\begin{document}


\dedication{%Für die armen Tieren auf dieser Welt.
Für Sissie.
}
\pdfbookmark{Startseite}{title}\maketitle


\pdfbookmark{\abstractname}{abstract}
\begin{abstract}
Die Software "`fineshift SLP-Rollout"' dient dem Ausrollen von Standardlastprofilen, angelehnt an die Praxisinformation P 2006/8 des BGW \cite{bgwp20068}.

Grundlage ist eine Access"=Datenbank mit allen nötigen Daten wie Jahresmengen der Kunden, Wet\-ter\-sta\-tio\-nen, Standardlastprofilparametern usw. usf.
\end{abstract}


\pdfbookmark{\contentsname}{toc}\tableofcontents
%\clearpage


%\begin{enumerate}
% \item Erster Block. Hier wird die Marke gesetzt.\label{verweis}
% \item Mittels \pageref{verweis} kann auf die Seitennummer des fraglichen Labels verwiesen werden.
% \end{enumerate}


% \vspace*{10cm}
 % \begin{longtable}{|l|r|c|p{2cm}|}
 % \hline
 % Linksbündige Spalte.&Rechtsbündige Spalte&Zentrierte Spalte&Parbox\\
 % \hline
 % Kurzer Text.&Kurzer Text.&Kurzer Text.&Kurzer Text.\\
 % \hline
 % Text.&Text.&Text.&In diesem Felde nun ein elendslanger text, um Umbrüche innerhalb eines Feldes zu erzeugen.\\
 % \hline
 % Text.&Text.&Text.&der Befehl $\backslash$vspace*\{\} weiter oben im Quellcode hat einem Umbruch vor dieser Zeile bewirkt.\\
 % \hline
 % \end{longtable}


% \begin{enumerate}
% \item Zweiter Block. Hier wird nach \vref{verweis} verwiesen.
% \item Mittels \pageref{verweis} kann auf die Seitennummer des fraglichen Labels verwiesen werden.
% \end{enumerate}


% \section{Erster Abschnitt}\label{sec:firstsection}
% Wenn $a=2$ und $b=3$ dann gilt:
% \begin{equation}\label{eq:simpleeq}
% a+b =2+3=5
% \end{equation}
% Wie in \cref{eq:simpleeq} gezeigt \ldots
% \newpage
% \section{Zweiter Abschnitt}\label{zweiter Abschnitt}
% \Cref{eq:simpleeq}\vpageref{eq:simpleeq} verdeutlichte bereits \ldots


% So sind 96 \% der bundesdeutschen Schulen mit Computern ausgestattet. \cite[S. 7]{bmbf} ...


%\chapter{Übersicht}\label{chap:overview}

Dieses Kapitel bietet eine Übersicht über das System.

\section{Einleitung}

Das Specios ist ein universelles Arten-System, d.\,h. es definiert Verhaltensregeln und legt Handlungsnotwendigkeiten fest, die erforderlich sind, um Wohlstand und Gleichberechtigung für jedes seiner Mitglieder sicherzustellen. Der Kern ist unabhängig von der Art des Lebewesens oder dessen Aufenthaltsort, denn wie könnte ein System jemals als "perfekt" angesehen werden, wenn es ausschließlich für den Menschen, die Erde bzw. unsere derzeitige Sicht auf die Gesellschaft entworfen würde?

\begin{quote}
"`Es ist kein Zeichen von Gesundheit, an eine zutiefst kranke Gesellschaft gut angepasst zu sein."'\\
--- Jiddu Krishnamurti
\end{quote}

Specios basiert auf aktuellen wissenschaftlichen Erkenntnissen über Gott und die Welt und bietet ein robustes Fundament. Es wird computergestützt arbeiten, bietet jedoch Notpläne, die einem Missbrauch entgegenwirken. Die Daten, die Specios benötigt, werden demokratisch von allen Mitgliedern bestimmt, wobei Kompetenzen berücksichtigt werden. Mit Hilfe des Internets ist es jederzeit offen für das ganze Volk - genau das ist der Trumpf und unsere einzige Gelegenheit aus den Fängen des Kapitalismus zu entweichen. Es ist möglich, durch vorübergehenden Verzicht auf übertriebenen Luxus, kurzfristig ein Mindest-Wohlstandslevel zu erreichen.

\begin{quote}
"`Die Welt hat genug für jedermanns Bedürfnisse, aber nicht für jedermanns Gier."'\\
--- Mahatma Gandhi
\end{quote}

Langfristig muss man weg kommen von der Profitmaximierung, man muss das tun, was wichtig ist für die Ziele des Specios, und damit im Endeffekt das Beste für jedes einzelne Mitglied. Es dürfen keine egoistischen Machtstellungen möglich sein. Das Venus-Projekt zeigt beispielsweise auf, dass wir technisch in der Lage sind, viel mehr Wohlstand zu erzeugen - mehr als ihn die wohlhabendsten Menschen heute haben. Unser aktuelles politisches System, zusammen mit dem Geldsystem, hindert uns jedoch gewaltig, da viele Zielsetzungen per Definition widersprüchlich sind. Durch eine repräsentative Darstellung der von Specios berechneten Notwendigkeiten, basierend auf grundlegenden Daten die von den Mitgliedern geliefert werden, muss sich die aktuelle Politik früher oder später danach richten.

\section{Weitere Tabellen}



%\verb|++|
%(siehe \vref{sec:rollout_mathematics}).

%\begin{figure}
%\caption[Überblick/Datenbank]{Überblick über die vom Programm verwendeten Tabellen}
%\includegraphics[width=\textwidth]{gfx/overview_database.png}
%\label{fig:overview_database}
%\end{figure}
\chapter{Einführung}\label{chap:introduction}

Dieses Kapitel wird den Hintergrund und die Notwendigkeit von einem grundlegend neuen Gesellschaftssystem verdeutlichen.

Verschiedenste Probleme sind allgemein bekannt:
\medskip
\begin{compactitem}
\item Hunger, Verschmutzung, niedrige Lebensqualität
\item Tierquälerei (Massentierhaltung mit allen Konsequenzen, Fischerei\footnote{Fische haben ein komplexes Schmerzbewusstsein analog zu Säugetieren, wie Studien der Gehirnaktivität und des Folgeverhaltens belegen. \citep{spiegel_10_2011}}, Pelz"=Beschaffung, überflüssige Tierversuche, Missbrauch für Unterhaltungszwecke)
\item Zerstörung der Natur (Abholzung, Ausrottung, Freisetzung fossiler CO\textsubscript{2}"=Einlagerungen, chemische und atomare Kontaminierung)
\item tödliche Auseinandersetzungen (Kriege, Terror"=Anschläge)
\item niedere Arbeiten
\item Überbevölkerung
\end{compactitem}
\medskip

Ein großer Fehler bei den meisten bisherigen Versuchen einer entsprechenden Problemlösungsstrategie oder zunächst einmal einer Ursachensanalyse innerhalb der Gesellschaftsordnung war der, nicht naturwissenschaftlich an das Problem heranzugehen; geblendet von Religionen und klassischen Ideologien. Es fängt bei der Argumentation an, was denn das Ziel für die Gesellschaft und den einzelnen Menschen darstellen soll, welche nur auf Grundlage willkürlicher Moralvorstellungen oder überhaupt nicht geführt wurde.

\section{Situation}\label{sec:situation}

Dieser Abschnitt wird die derzeitige Gesellschaftssituation verdeutlichen, warum wir hier gelandet sind, warum es scheinbar ganz gut funktioniert, warum bisherige Änderungsversuche scheiterten und warum das trotzdem nicht so bleiben kann.

\subsection{Kapitalismus}\label{sec:situation/capitalism}

Wir leben im Kapitalismus. Selbst sozialistische Staaten wie China, Nordkorea, Vietnam oder Kuba sind in die globale Marktwirtschaft eingebunden und eigentlich eher kapitalistisch. Zumindest kann nicht bestritten werden, dass jeder irgendwie nach Geld giert.

Wird man davon glücklich? In unserem System ein klares ja. Gesundheit, Spaß, Zeit. Alles kostet Geld und die meiste Arbeit heute dient einzig der Vermehrung von selbigem. Wenn man das Tauschmittel Geld nur oberflächlich betrachtet, macht es natürlich Sinn: Als Jäger und Sammler konnten wir noch eigentumslos leben, alles wurde in der Gruppe aufgeteilt. Später entwickelte man aber den Ackerbau und die Viehzucht, wurde sesshaft. Bauern waren in der Lage mehr Nahrung zu produzieren als sie selbst benötigten, ihnen fehlten dafür aber andere Dinge; der Tauschhandel begann. Um die Suche nach einem Tauschpartner zu vereinfachen, die teilweise geringe Haltbarkeit der Tauschgüter zu umgehen und die Frage nach dem Gegenwert zu verallgemeinern, wurden später möglichst seltene Zwischentauschmittel eingeführt, wie z.\,B. Kaurischneckenhäuser in China. In der Folge wurden daraus Münzen und schließlich unser heutiges Geld. Das klingt zunächst wie eine perfekte Lösung.

Ein großes Problem ist die Verwaltung des Geldes. Heutzutage wird es vermehrt ohne einen echten Gegenwert zu haben. Durch das Zinssystem gibt es immer mehr Schulden als (virtuelles) Geld, d.\,h. die ganze Welt kann den privatisierten Banken gegenüber insgesamt niemals schuldenfrei sein. Großkonzerne und Banken haben Geld und demnach Macht über jeden, der Geld benötigt. Nicht nur normale Bürger, auch die Regierungen gehören dazu. Folglich bringt es rein garnichts in die Politik zu gehen, wenn man etwas daran ändern möchte -- die sogenannte "`Elite"', wie sie Verschwörungstheoretiker nennen, wird das verhindern. Das ist leider keine Verschwörungstheorie sondern eine Tatsache. Die genauen Verhältnisse kann man selbst überall nachlesen.

Ein weiteres entscheidendes Problem ist relativ jung. Wir sind an einem Punkt angelangt, an dem wir die gesamte Welt problemlos versorgen könnten.\footnote{Siehe dazu auch die Ergebnisse des Venus"=Projektes} Doch alles was man zum Leben benötigt kostet Geld und Geld bekommt man nur durch Arbeit. Wie schafft man sich Arbeit, wenn es eigentlich kaum welche geben muss? Man produziert billig und kurzlebig. Man sorgt künstlich für Knappheit, denn was selten ist, ist wertvoll. Beispielsweise wurden in der Kimberly Diamantenmine Diamanten verbrannt, um den Preis hoch zu halten. Darüberhinaus lässt man monotone und anspruchslose Arbeit von Menschen statt von Maschinen erledigen, um die Arbeitslosenzahlen niedrig zu halten, denn man braucht ja Arbeit, um zu leben -- selbst wenn sie vollkommen sinnlos ist. Effizienz, Reichhaltigkeit und Nachhaltigkeit widersprechen folglich der Struktur des profitorientierten Systemes.

\subsection{Gesetze}\label{sec:situation/laws}

Jedes Lebewesen ist im Kern egoistisch, das ist biologisch überlebensnotwendig. Allerdings muss es mit der Umwelt auf gewisse Weise kooperieren, wenn es überleben will\footnote{\textit{Überleben} als obersten Willen darzulegen ist eigentlich falsch, genügt aber hier. Das tatsächliche Lebensziel wird in \vmyref{sec:basis/aim} behandelt.}. Folglich muss sich ein Organismus einschränken und kontrollieren, um mit anderen erfolgreich zu interagieren. Diese symbiotischen Beziehungen werden komplexer, je wirkungsvoller und weitreichender (auch zeitlich) sie ausgelegt sind und können von einem Organismus ohne weiteres nicht vollständig überschaut werden. Daher sind (global) einzuhaltende Regeln erforderlich, um letzlich das eigene Wohl sicherzustellen, weswegen sich auch Grundbedürfnisse und Instinkte ausgebildet haben. Daher haben sich unsere Vorfahren in Rudeln organisiert.

\subsection{Demokratie}\label{sec:situation/democracy}

Es wird allgemein behauptet, dass wir in einer Demokratie leben. Tatsächlich haben wir aber kaum Einfluss auf Entscheidungen. Wir wählen Parteien, welche anschließend machen was sie wollen. Wäre dies nicht der Fall und würden sie für das Allgemeinwohl des Volkes arbeiten, würde nicht so viel Elend in der Welt herrschen, das genügt hier als Beweis.

Der Sinn dahinter war, dass Entscheidungen heruntergebrochen werden um die Komplexität zu verringern. Allerdings gehen übergeordnete Instanzen genauso wenig auf untergeordnete ein, wie Partien auf die Bürger eingehen. Hier und da gibt es zwar Regeln, allerdings werden sie immer weiter aufgeweicht, hauptsächlich mit Hilfe psychologischer Manipulation, wodurch untergeordnete Instanzen, vor allem normale Bürger, diversen Initiativen quasi aus dem falschen Bauchgefühl heraus zustimmen. Dazu zählt insbesondere der Terror"=Hype. Seine Folgen sind beispielsweise die Lockerung des Datenschutzes inkl. steigender Überwachung oder die Aushebelung des deutschen Gesetzgebungsverfahrens durch den Vertrag von Lissabon.

Allzu direkt darf die Demokratie jedoch auch nicht sein. Als einfacher Bürger kann man die Komplexität hinter den Entscheidungen garnicht überblicken und handelt oft voreingenommen. Die Aufgabe der Parteien und Lobbyisten ist es, komplizierte Sachverhalte zu analysieren und Lösungen mit entsprechenden Begründungen vorzuschlagen, welche dann akzeptiert werden können oder nicht. Entscheidungen beruhen also auf dem Vertrauen weniger Instanzen, welche dieses natürlich zu ihren Gunsten ausnutzen können.

\subsection{Analyse}\label{sec:situation/analysis}

Wir sind Rudeltiere; evolutionstechnisch ist es nicht abwegig Fremde zunächst als Feinde zu betrachten. Folglich wirkt Altruismus höchstens in einem Familien"=Clan.

Selbstlosigkeit bzw. Solidarität ist also nicht angeboren; Gier wird dagegen aber nur in einer profitorientierten oder ressourcen"=armen Gesellschaft notwendig. Ersteres bedingt der Kapitalismus und letzteres wird für ihn konstruiert, wie wir oben festgestellt haben.

Das Wohl eines Menschen darf nicht davon abhängen, dass er Probleme löst, denn sonst erzeugt er sie künstlich (wie Krieg oder Nahrungsmittelvernichtung).

Unser System weist grundlegende Widersprüche auf und Änderungen im benötigen Maßstab sind so nicht möglich. Man muss ganz von vorne beginnen. Wir können aber nicht ewig darauf warten, in der Hoffnung, dass uns unser "`Fortschritt"' schon irgendwann zu einem besseren System befähigt. Denn einerseits sind wir technologisch so gut wie festgefahren und andererseits werden wir immer weiter eingeschränkt, so dass wir uns irgendwann nicht mehr wehren können. Wir sind schon jetzt Sklaven, aber noch arbeiten wir unter freiem Himmel.

Mit den politischen Mitteln, die uns zur Verfügung gestellt werden, kommen wir nicht weiter. Die bloße Verteilung von aufklärenden Informationen genügt aber auch nicht: Die an den wichtigen Positionen zu wohlhabende und allgemein müde Masse beginnt nicht auf Basis von systemfeindlichen Diskussionen, sich rückhaltslos dem gewöhnlichem Alltag zu widersetzen, geschweige denn kollektiv und ausreichend konsequent. 

\section{Konsequenzen}\label{sec:consequences}

In diesem Abschnitt werden die Konsequenzen aus der in \vref{sec:situation} dargelegten Situations-Analyse gezogen, welche die notwendigen Anforderungen an die Gesellschaftsrevolution definieren.

\subsection{Anforderungen}\label{sec:consequences/requirements}

Folgende Anforderungen müssen erfüllt werden, um einen erfolgreichen Wandel herbeizuführen:
\begin{itemize}
\item Massive Verhaltensänderungen oder massive Proteste gegen gegenwärtige Ideologien müssen organisiert und konsistent ablaufen.
\item Es muss viel bessere Aufklärungsarbeit geleistet werden -- flächendeckend, logisch und einfach nachvollziehbar.
\item Die Regeln müssen jederzeit von jedem nachvollziehbar sein und in Frage gestellt werden können.
\item kein Mensch oder anderes Lebewesen darf im Mittel mehr Einfluss haben als die anderen.
\end{itemize}

\subsection{Synthese}\label{sec:consequences/synthesis}

Aus den Anforderungen folgt folgende notwendige Ideologie:
\begin{itemize}
\item Jegliche Meinungen der Menschen müssen in Relation zu ihrer Kompetenz auf dem themasierten Gebiet betrachtet werden und das System direkt beeinflussen.
\item Man benötigt ein Werkzeug, welches (komplexe) Zusammenhänge sowohl verdeutlicht als auch verifiziert.
\item Das gesamte System muss transparent über ein überall verfügbares Medium erfassbar sein.
\item Konsistenz, Korrektheit und Nachvollziehbarkeit muss durch Selbstdefinition der Regeln anhand so elementarer Entscheidungen wie möglich gewährleistet werden. 
\end{itemize}

Es ist also ein (wenn man so will im wahrsten Sinne des Wortes diktatorisches) Computersystem erforderlich, dessen "`Gehirn"' von der Menschheit demokratisch aufgebaut wird und dadurch nicht selbst korrupt werden kann. Sämtliche abstrahierten/""finalen Entscheidungen sind demnach architektonisch bedingt von der Bevölkerung gedeckt und können jederzeit durch Argumentationsarbeit (Erweiterung/""Korrektur des Wissens des Systemes) angepasst werden. Zu den Anpassungen zählen nicht nur einfache Daten sondern auch Formeln, welche Zusammenhänge definieren.

Um nicht am Konservatismus zu scheitern muss eine möglichst vollständige Zielvorstellung vorgelegt werden. Dies kann am effektivsten mit dem Aufbau eines virtuellen \myindexednote{Parallelsystem}Parallelsystemes erfüllt werden, welches die Vorteile und Wege klar präsentiert und somit auf Basis der Einsicht und Akzeptanz fließend angewendet werden kann. Die herrschende Minderheit wird auf diese Weise moralisch gezwungen sein, die Weichen entsprechen zu stellen und schließlich gewaltlos entmächtigt.

\chapter{Konzept}\label{chap:concept}

Dieses Kapital wird eine Lösung skizzieren, welche die in \vref{sec:consequences} dargelegten Anforderungen erfüllt.

Ich nennen das System \textit{Specios, la specio sistemo}\footnote{"`Specios, la specio sistemo"' bedeutet "`Specios, das Arten-System"', wobei \textit{Art} im Sinne von \textit{Spezies} zu verstehen ist, und \textit{System} u.\,A. im politischen Sinne. Also ein Verwaltungssystem für jede Art von Lebewesen. Wenn man möchte, kann man \textit{Specios} selbst auch ähnlich eines rekursiven Akronyms betrachten, es steht für "`\textbf{Specio} \textbf{s}ystemo por \textbf{s}inkonservo, \textbf{p}rospero, \textbf{e}galrajteco kaj la \textbf{c}elado al \textbf{i}dentigado de \textbf{o}bskureco"' (epo.\index{Esperanto} Arten-System zur Selbsterhaltung, Wohlstand, Gleichberechtigung und dem Bestreben nach der Identifizierung der Unbekanntheit). Zu beachten ist, dass "`specios"' nicht esperanto ist, bezeichnet also auch nicht die Mehrzahl von "`specio"', dies wäre "`specioj"'.}.

\section{Grundlage}\label{sec:basis}

\subsection{Ziel}\label{sec:basis/aim}

Nachdem man einsehen muss, dass es keinen Gott und keine andere höhere Macht gibt, welche mit uns ein Ziel verfolgen könnte, und auch die Existenz eines Geistes nicht erforderlich ist, um unser Dasein und Handeln zu erklären, bleibt uns nicht mehr viel. Da nicht nur unser Gehirn, sondern selbst eine einzelne lebende Zelle physikalisch extrem schwer erfassbar ist und möglicherweise diverse Quanteneffekte eine Rolle spielen, die wir noch garnicht vollständig, zumindest in der Gesamtheit unseres Organismus, verstehen, dürfen wir uns aber nicht anmaßen, dort grob fahrlässig eingreifen zu dürfen.

Unter Beobachtung der Entwicklung aller lebenden Wesen können wir relativ sicher eine Gemeinsamkeit feststellen, welche man zum Ziel bzw. Recht erheben kann: Die Fortpflanzung seiner selbst bzw. den Erhalt seiner Spezies. Welches von beiden das jeweils andere impliziert, spielt dabei keine Rolle. Mit dieser Grundlage kann man alle sozialen Strukturen erklären (nicht nur menschliche) und natürlich auch eine bessere ausarbeiten, welche Zwecks Robustheit möglichst allgemeingültig und systematisch zu formulieren ist.

Aufgrund der prinzipiell erlaubten Willkür, können wir unter Einhaltung erwähnter Grenzen allen Mitgliedern, die sich an die festgelegten Regeln halten, von aktiver Schädigung befreien und gegen die Herstellung maximaler Zufriedenheit streben. Je nach genetischer Verwandtheit werden die Rechte auf andere Lebensformen entsprechend übertragen, d.h. insbesondere, dass auch Tiere ein Recht auf Wohlstand haben. Ob es erlaubt sein wird, sie zu töten um sie zu verspeisen, hängt von vielen Wichtungsfaktoren ab, welche, wie alles andere auch, mit Hilfe des kompetenzgewichtet-demokratischen Wahlsystemes festgelegt werden. Dabei werden von den Mitgliedern möglichst elementare Entscheidungen getroffen und nach oben zu allgemeineren Schlussfolgerungen synthetisiert.

\section{Regelwerk}\label{sec:rulebook}

\chapter{System-Architektur}

In diesem Kapitel wird die praktische Umsetzung des in \vref{chap:concept} beschriebenen Konzeptes von Specios betrachtet.

\section{Verwaltung}

\section{Wahlsystem}

ich glaub ich mach das so:
1) daten oder datenblöcke (intern dasselbe, aber der verbund muss wiedererkennbar sein, um blockweise wahlen zu ermöglichen) über alles mögliche, aber immer mit orts-, zeitraum- und genauigkeits-angaben können von jedem benutzer über spezielle webinterfaces oder über eine API eingetragen werden. jeder benutzer kann daten über dieselbe sache in derselben orts- und zeit-ebene nur 1x im system haben, darf sich auch nicht überschneiden. wenn ein benutzer die daten einer anderen person wählt, ist das logisch (und vielleicht auch technisch) dasselbe als würde er eben diese daten von sich aus einfügen.
2) auf magische weise werden dann mehrheits- und kompetenzgewichtungen reingeworfen, überschneidungen und konflikte aufgelöst (besonders lustig bei zahlendaten) und am ende stehen dann die besten vorschläge da.
3) in einem schritt, vielleicht demselben, werden dann konflikte zwischen abstraktionsschichten irgendwie aufgelöst, diese überhaupt zu finden ist vielleicht auch nicht gerade einfach. und untere schichten müssen auf alle oberen vereinfacht werden.
4) ein problem ist bei dem ganzen kram, dass bestimmte operationen vorzugsweise unmittelbar vom system bearbeitet werden müssen, z.b. übersetzungen von wörtern müssen sofort übernommen werden, das einfügen neuer daten sollte sofort sichtbar sein, usw.
5) und dann kommt die richtige magie, lauter demokratisch ermittelte scripte, die zusammenhänge zwischen den rohdaten herstellen und höhere tatsachen definieren. z.b. die wichtigkeit der spezies mücke bezogen auf dessen intelligenz und der anzahl lebender exemplare in deutschland oder so. diese scripte generieren also neue daten, die die "richtigkeit" der baiserenden daten mit einbeziehen muss, damit falls sie neue daten definieren für die es bereits manuelle meinungen gibt, wieder neu entschieden werden kann was denn nun korrekt ist - klingt kompliziert.
6) und zum schluss muss specios aktions-skripte und dessen auswirkungen über die zeit simulieren, um zu ermitteln wie am ende die basisziele maximal erfüllt werden. für einen gedanken muss ggf. alles ab punkt 3 wiederholt werden. ungenaue aber schneller ableitbare folgen mit entsprechend schlechteren wahrscheinlichkeiten erhöhen dabei die arbeitsgeschwindigkeit, also muss der kompromisse zwischen zeit, genauigkeit und wichtigkeit fällen.
klingt das machbar?

\section{Künstliche Intelligenz}

ich glaube die implementierung von specios würde einer ki wie ich/falk sie bisher halbwegs geplant haben ziemlich nahe kommen müssen:
z.b. muss es informationen verallgemeinern können (aus effizienzgründen), abschätzen welchen fehler informationen haben und ggf. spezialisieren/neu berechnen. dafür muss es aber hoffentlich nicht unbedingt informationen in zusammenhang setzen oder neue informationen ableiten, das ist ja mehr oder weniger der demokratische akt: meinungen sammeln, zusammenhänge aufbauen, gesetze formulieren. das system zeigt quasi nur auf abstrakter weise auf was sache ist und was getan werden muss.
und es muss aktivitäten simulieren, was auswirkungen auf sämtliche daten hat und erkenntnisse zulässt.
das hauptproblem neben der spezifikation ist vermutlich der denkvorgang mit dem beschränkten arbeitsspeicher.

\section{Sicherheit}

\chapter{Sonstiges}\label{chap:misc}

\section{Geschichte}\label{sec:history}

\begin{itemize}[zwischen 2003 und 2006:]
\item[zwischen 2003 und 2006:] Initiale Ideensammlung zum perfekten politischen System.
\item[15.03.2007:] Die Idee zu DADD (Demokratische AutoDiktatur Deutschland)\footnote{Urheber will nicht genannt werden.} ist entstanden.
\item[25.03.2007:] Die Idee zu Specios ist entstanden.
\item[19.07.2010:] Erste Konzept"=Formulierung.
\item[ab August 2010:] Webseitenentwicklung und nähere Analysen zur technischen Umsetzung.
\end{itemize}
\section{Meilensteine}\label{sec:milestones}

\begin{enumerate}[1.]
\item Ausarbeitung des Konzeptes.
\item Strukturiertes System zur Erstellung und Bewertung von Inhalten.
\item Automatische Ableitung von höheren Zielen, Regeln u.\,ä. anhand der elementaren Entscheidungen.
\item Zuverlässige Repräsentation des Meinungsbildes der Bevölkerung.
\item Adaption der Politik an diese Regeln.
\item Stabiles und sicheres Konzept zur Wahrung des Supera Specios.
\end{enumerate}

\section{Schlusswort}\label{sec:conclusion}

Im Endeffekt erhält man mit Specios ein mögliches System nach dem man sich richten kann, oder eben nicht. Man wird jedoch feststellen, dass sein Wohlbefinden durch dieses System am besten sichergestellt werden kann und es deshalb jedem anderen vorziehen -- auch wenn man die o.\,g. Grundlagen noch nicht ganz unterschreiben möchte.

\appendix


\listoffigures
\listoftables
\printindex

%\begin{thebibliography}{BGW/VKU, 2007}
   %\bibitem[BGW, 2006]{bgwp20068}\textsc{Bundesverband der deutschen Gas- und Wasserwirtschaft} (2006): \textit{Leitfaden SLP. Praxisinformation P 2006/8 Gastransport/Betriebswirtschaft: Anwendung von Standardlastprofilen zur Belieferung nicht"=leistungsgemessener Kunden.}
   %\bibitem[BGW/VKU, 2007]{bgwvkup200713}\textsc{Bundesverband der deutschen Gas- und Wasserwirtschaft; Verband kommunaler Unternehmen e. V.} (2007): \textit{Praxisinformation P2007/13 Gastransport/Betriebswirtschaft: Abwicklung von Standardlastprofilen.}
   % \bibitem[BMBF, 2003]{bmbf}"'IT-Ausstattung der allgemein bildenden und berufsbildenden Schulen in Deutschland"', http://www.schulen-ans-netz.de/neuemedien/fakten/dokus/it-ausstattung-2003.pdf, 10.03.2005
%\end{thebibliography}
\bibliography{main}

\end{document} 