\section{Konsequenzen}\label{sec:consequences}

In diesem Abschnitt werden die Konsequenzen aus der in \vref{sec:situation} dargelegten Situations-Analyse gezogen, welche die notwendigen Anforderungen an die Gesellschaftsrevolution definieren.

\subsection{Anforderungen}\label{sec:consequences/requirements}

Folgende Anforderungen müssen erfüllt werden, um einen erfolgreichen Wandel herbeizuführen:
\begin{itemize}
\item Massive Verhaltensänderungen oder massive Proteste gegen gegenwärtige Ideologien müssen organisiert und konsistent ablaufen.
\item Es muss viel bessere Aufklärungsarbeit geleistet werden -- flächendeckend, logisch und einfach nachvollziehbar.
\item Die Regeln müssen jederzeit von jedem nachvollziehbar sein und in Frage gestellt werden können.
\item kein Mensch oder anderes Lebewesen darf im Mittel mehr Einfluss haben als die anderen.
\end{itemize}

\subsection{Synthese}\label{sec:consequences/synthesis}

Aus den Anforderungen folgt folgende notwendige Ideologie:
\begin{itemize}
\item Jegliche Meinungen der Menschen müssen in Relation zu ihrer Kompetenz auf dem themasierten Gebiet betrachtet werden und das System direkt beeinflussen.
\item Man benötigt ein Werkzeug, welches (komplexe) Zusammenhänge sowohl verdeutlicht als auch verifiziert.
\item Das gesamte System muss transparent über ein überall verfügbares Medium erfassbar sein.
\item Konsistenz, Korrektheit und Nachvollziehbarkeit muss durch Selbstdefinition der Regeln anhand so elementarer Entscheidungen wie möglich gewährleistet werden. 
\end{itemize}

Es ist also ein (wenn man so will im wahrsten Sinne des Wortes diktatorisches) Computersystem erforderlich, dessen "`Gehirn"' von der Menschheit demokratisch aufgebaut wird und dadurch nicht selbst korrupt werden kann. Sämtliche abstrahierten/""finalen Entscheidungen sind demnach architektonisch bedingt von der Bevölkerung gedeckt und können jederzeit durch Argumentationsarbeit (Erweiterung/""Korrektur des Wissens des Systemes) angepasst werden. Zu den Anpassungen zählen nicht nur einfache Daten sondern auch Formeln, welche Zusammenhänge definieren.

Um nicht am Konservatismus zu scheitern muss eine möglichst vollständige Zielvorstellung vorgelegt werden. Dies kann am effektivsten mit dem Aufbau eines virtuellen \myindexednote{Parallelsystem}Parallelsystemes erfüllt werden, welches die Vorteile und Wege klar präsentiert und somit auf Basis der Einsicht und Akzeptanz fließend angewendet werden kann. Die herrschende Minderheit wird auf diese Weise moralisch gezwungen sein, die Weichen entsprechen zu stellen und schließlich gewaltlos entmächtigt.
