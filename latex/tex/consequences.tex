\section{Konsequenzen}\label{sec:consequences}

In diesem Abschnitt werden die Konsequenzen aus der in \vref{sec:situation} dargelegten Situations-Analyse gezogen, welche die notwendigen Anforderungen an die Gesellschaftsrevolution definieren.

\subsection{Revolutions-Anforderungen}\label{sec:consequences/requirements}

[*]Wir wollen ein besseres System.
[*]Wir können nicht auf eine Revolution/Besserung warten, da uns die Freiheit, eine solche zu realisieren, immer weiter genommen wird - warten wäre zu riskant. Außerdem ist der Auslöser dafür nicht zu erwarten: Man will seine Sklaven schließlich behalten! Sie werden uns füttern, gerade so viel, dass wir arbeiten können und mangels Alternativen auch arbeiten werden.
[*]Wählen genügt nicht; grundsätzliche Gegensätze lassen sich nicht mit Systemmitteln korrigieren.
[*]Die Selbstinitiative der Menschen ist zu niedrig, da es Menschen in günstigen Positionen zu gut geht und Verhaltensänderungen gegen dieses System existenzgefährdend sind und keiner glaubt, dass er etwas erreichen würde. Dies muss aber in Massen geschehen. Folglich müssen wir viel bessere Aufklärungsarbeit leisten - flächendeckend, logisch und einfach nachvollziehbar.
[*]Nachfolgende massive Verhaltensänderungen oder massive Proteste gegen gegenwärtige Ideologien müssen organisiert und konsistent ablaufen, um zu wirken.
[*]Absolute Demokratie führt zu Chaos.
[*]Die Organisation bzw. der Aufruf zu entsprechenden Tätigkeiten darf nicht von einer Minderheit ausgehen (Diktatur-Gefahr).

Wichtig dabei ist die Möglichkeit, die Regeln jederzeit nachvollziehen und in Frage stellen zu können.

\subsection{Synthese}\label{sec:consequences/synthesis}

Am effektivsten ist es, ein zunächst virtuelles Parallelsystem aufzubauen, welches die Vorteile und Wege klar präsentiert und somit auf Basis der Einsicht fließend angewendet werden kann. Die herrschende Minderheit wird auf diese Weise moralisch gezwungen sein, die Weichen entsprechen zu stellen und schließlich gewaltlos entmächtigt.

Folglich gibt es nur eine einzige Lösung: Schaffung eines besseren Parallelsystemes, welches mittels Akzeptanz nach und nach von selbst übernommen wird. Um obige Aspekte zu berücksichtigen, muss es folgende Merkmale aufweisen:

Dies ist die direkte Folge der Komplexität der gegenseitigen Beziehungen, wie eingangs erwähnt. Man benötigt also ein Werkzeug, welches die Zusammenhänge sowohl verdeutlicht als auch verifiziert. Darüber hinaus müssen jegliche Meinungen der Menschen in Relation zu ihrer Kompetenz auf dem themasierten Gebiet betrachtet werden.

[*]Transparent über das Internet erfassbar.
[*]Kontrolliert von jedem einzelnen Menschen unter Berücksichtigung seiner Kompetenzen.
[*]Konsistent, korrekt und nachvollziehbar durch Selbstdefinition anhand so elementarer Entscheidungen wie möglich. Es ist also ein (wenn man so will im wahrsten Sinne des Wortes diktatorisches) Computersystem erforderlich, dessen "Gehirn" von der Menschheit demokratisch aufgebaut wird und dadurch nicht korrupt werden kann. Sämtliche höherwertigen/finalen Entscheidungen sind demnach architektonisch bedingt von der Bevölkerung gedeckt und können jederzeit durch Argumentationsarbeit (Erweiterung/Korrektur des Wissens des Systemes) angepasst werden.

Und zwar mit Hilfe eines kompetenzgewichteten Wahlsystemes über grundlegende Dinge und Daten. Daraus baut das System mit Hilfe von ebenfalls gewählten Formeln diverse Zusammenhänge auf, welche mit Hilfe eines großen Teiles einer KI möglichst effektiv, d.h. in einem Kompromiss zwischen Prioritäten, Zeit und Abstraktionsebene, die Folgen möglicher Tätigkeiten simuliert und somit notwendige Handlungen berechnet, welche nach und nach verfeinert und optimiert werden.