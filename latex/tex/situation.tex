\section{Situation}\label{sec:situation}

Dieser Abschnitt wird die derzeitige Gesellschaftssituation verdeutlichen, warum wir hier gelandet sind, warum es scheinbar ganz gut funktioniert, warum bisherige Änderungsversuche scheiterten und warum das auf keinen Fall so bleiben darf.

\subsection{Kapitalismus}\label{sec:situation/capitalism}

Wir leben im Kapitalismus. Selbst sozialistische Staaten wie China, Nordkorea, Vietnam oder Kuba sind in die globale Marktwirtschaft eingebunden und eigentlich eher kapitalistisch. Zumindest kann nicht bestritten werden, dass jeder irgendwie nach Geld giert.

Wird man davon glücklich? In unserem System ein klares ja. Gesundheit, Spaß, Zeit. Alles kostet Geld und die meiste Arbeit heute dient einzig der Vermehrung von selbigem. Wenn man das Tauschmittel Geld nur oberflächlich betrachtet, macht es natürlich Sinn: Als Jäger und Sammler konnten wir noch eigentumslos leben, alles wurde in der Gruppe aufgeteilt. Später entwickelte man aber den Ackerbau und die Viehzucht, wurde sesshaft. Bauern waren in der Lage mehr Nahrung zu produzieren als sie selbst benötigten, ihnen fehlten dafür aber andere Dinge; der Tauschhandel begann. Um die Suche nach einem Tauschpartner zu vereinfachen, die teilweise geringe Haltbarkeit der Tauschgüter zu umgehen und die Frage nach dem Gegenwert zu verallgemeinern, wurden später möglichst seltene Zwischentauschmittel eingeführt, wie z.\,B. Kaurischneckenhäuser in China. In der Folge wurden daraus Münzen und schließlich unser heutiges Geld. Das klingt zunächst wie eine perfekte Lösung.

Ein großes Problem ist die Verwaltung des Geldes. Heutzutage wird es vermehrt ohne einen echten Gegenwert zu haben. Durch das Zinssystem gibt es immer mehr Schulden als (virtuelles) Geld, d.\,h. die ganze Welt kann den privatisierten Banken gegenüber insgesamt niemals schuldenfrei sein. Großkonzerne und Banken haben Geld und demnach Macht über jeden, der Geld benötigt. Nicht nur normale Bürger, auch die Regierungen gehören dazu. Folglich bringt es rein garnichts in die Politik zu gehen, wenn man etwas daran ändern möchte -- die sogenannte "`Elite"', wie sie Verschwörungstheoretiker nennen, wird das verhindern. Das ist nur leider keine Verschwörungstheorie sondern eine Tatsache. Die genauen Verhältnisse kann man überall selbst nachlesen.

Ein weiteres entscheidendes Problem ist relativ jung. Wir sind an einem Punkt angelangt, an dem wir die gesamte Welt problemlos versorgen könnten.\footnote{Siehe dazu auch die Ergebnisse des Venus"=Projektes} Doch alles was man zum Leben benötigt kostet Geld und Geld bekommt man nur durch Arbeit. Wie schafft man sich Arbeit, wenn es eigentlich kaum welche geben muss? Man produziert billig und kurzlebig. Man sorgt künstlich für Knappheit, denn was selten ist, ist wertvoll. Beispielsweise wurden in der Kimberly Diamantenmine Diamanten verbrannt, um den Preis hoch zu halten. Darüberhinaus lässt man monotone und anspruchslose Arbeit von Menschen statt von Maschinen erledigen um die Arbeitslosenzahlen niedrig zu halten, denn man braucht ja Arbeit, um zu leben -- selbst wenn sie vollkommen sinnlos ist. Effizienz, Reichhaltigkeit und Nachhaltigkeit widersprechen folglich der Struktur des profitorientierten Systemes.

\subsection{Gesetze}\label{sec:situation/laws}

Jedes Lebewesen ist im Kern egoistisch, das ist biologisch überlebensnotwendig. Allerdings muss es mit der Umwelt auf gewisse Weise kooperieren, wenn es überleben will\footnote{\textit{Überleben} als obersten Willen darzulegen ist eigentlich falsch, genügt aber hier. Das tatsächliche Lebensziel wird in \vmyref{sec:basis/aim} behandelt.}. Folglich muss sich ein Organismus einschränken und kontrollieren, um mit anderen erfolgreich zu interagieren. Diese symbiotischen Beziehungen werden komplexer, je wirkungsvoller und weitreichender (auch zeitlich) sie ausgelegt sind und können von einem Organismus ohne weiteres nicht vollständig überschaut werden. Daher sind (global) einzuhaltende Regeln erforderlich, um letzlich das eigene Wohl sicherzustellen, weswegen sich auch Grundbedürfnisse und Instinkte ausgebildet haben. Daher haben sich unsere Vorfahren in Rudel überschauberer Größe organisiert.

\subsection{Demokratie}\label{sec:situation/democracy}

Es wird allgemein behauptet, dass wir in einer Demokratie leben. Tatsächlich haben wir aber kaum Einfluss auf Entscheidungen. Wir wählen Parteien, welche anschließend machen was sie wollen. Wäre dies nicht der Fall und würden sie für das Allgemeinwohl des Volkes arbeiten, würde nicht so viel Elend in der Welt herrschen, das genügt hier als Beweis.

Der Sinn dahinter war, dass Entscheidungen heruntergebrochen werden um die Komplexität zu verringern. Allerdings gehen übergeordnete Instanzen genauso wenig auf untergeordnete ein, wie Parteien auf die Bürger eingehen. Hier und da gibt es zwar Regeln, allerdings werden sie immer weiter aufgeweicht, hauptsächlich mit Hilfe psychologischer Manipulation, wodurch untergeordnete Instanzen, vor allem normale Bürger, diversen Initiativen quasi aus dem falschen Bauchgefühl heraus zustimmen. Dazu zählt insbesondere der Terror"=Hype. Seine Folgen sind beispielsweise die Lockerung des Datenschutzes inkl. steigender Überwachung oder die Aushebelung des deutschen Gesetzgebungsverfahrens durch den Vertrag von Lissabon.

Allzu direkt darf die Demokratie jedoch auch nicht sein. Als einfacher Bürger kann man die Komplexität hinter den Entscheidungen garnicht überblicken und handelt oft voreingenommen. Die Aufgabe der Parteien und Lobbyisten ist es, komplizierte Sachverhalte zu analysieren und Lösungen mit entsprechenden Begründungen vorzuschlagen, welche dann akzeptiert werden können oder nicht. Entscheidungen beruhen also auf dem Vertrauen weniger Instanzen, welche dieses natürlich zu ihren Gunsten ausnutzen können.

\subsection{Analyse}\label{sec:situation/analysis}

Wir sind Rudeltiere; evolutionstechnisch ist es nicht abwegig Fremde zunächst als Feinde zu betrachten. Folglich wirkt Altruismus höchstens in einem Familien"=Clan.

Selbstlosigkeit bzw. Solidarität ist also nicht angeboren; Gier wird dagegen aber nur in einer profitorientierten oder ressourcen"=armen Gesellschaft notwendig. Ersteres bedingt der Kapitalismus und letzteres wird für ihn konstruiert, wie wir oben festgestellt haben.

Das Wohl eines Menschen darf demnach nicht davon abhängen, möglichst viel Profit zu erwirtschaften oder allgemein gesagt Probleme zu lösen, denn sonst erzeugt er sie künstlich (wie Krieg oder Nahrungsmittelvernichtung).

Man muss also feststellen, dass unser System grundlegende Widersprüche aufweist und Änderungen im benötigen Maßstab sind so nicht möglich. Man muss ganz von vorne beginnen. Wir können aber nicht ewig darauf warten, in der Hoffnung, dass uns unser "`Fortschritt"' schon irgendwann zu einem besseren System befähigt. Denn einerseits sind wir technologisch so gut wie festgefahren und andererseits werden wir immer weiter eingeschränkt, so dass wir uns irgendwann nicht mehr wehren können. Wir sind schon jetzt Sklaven, aber noch arbeiten wir unter freiem Himmel.

Mit den politischen Mitteln, die uns zur Verfügung gestellt werden, kommen wir nicht weiter. Die bloße Verteilung von aufklärenden Informationen genügt aber auch nicht: Die an den wichtigen Positionen zu wohlhabende und allgemein müde Masse beginnt nicht auf Basis von systemfeindlichen Diskussionen, sich rückhaltlos dem gewöhnlichem Alltag zu widersetzen, geschweige denn kollektiv und ausreichend konsequent. 
