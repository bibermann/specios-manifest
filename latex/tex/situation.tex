\section{Situation}\label{sec:situation}

Dieser Abschnitt wird die derzeitige Gesellschaftssituation verdeutlichen, warum wir hier gelandet sind, warum es scheinbar ganz gut funktioniert, warum bisherige Änderungsversuche scheiterten und warum das auf keinen Fall so bleiben darf.

\subsection{Kapitalismus}\label{sec:situation/capitalism}

Wir leben im Kapitalismus. Selbst sozialistische Staaten wie China, Nordkorea, Vietnam oder Kuba sind in die globale Marktwirtschaft eingebunden und eigentlich eher kapitalistisch. Zumindest kann nicht bestritten werden, dass jeder irgendwie nach Geld giert.

Wird man davon glücklich? In unserem System ein klares ja. Gesundheit, Spaß, Zeit. Alles kostet Geld und die meiste Arbeit heute dient einzig der Vermehrung von selbigem. Wenn man das Tauschmittel Geld nur oberflächlich betrachtet, macht es natürlich Sinn: Als Jäger und Sammler konnten wir noch eigentumslos leben, alles wurde in der Gruppe aufgeteilt. Später entwickelte man aber den Ackerbau und die Viehzucht, wurde sesshaft. Bauern waren in der Lage mehr Nahrung zu produzieren als sie selbst benötigten, ihnen fehlten dafür aber andere Dinge; der Tauschhandel begann. Um die Suche nach einem Tauschpartner zu vereinfachen, die teilweise geringe Haltbarkeit der Tauschgüter zu umgehen und die Frage nach dem Gegenwert zu verallgemeinern, wurden später möglichst seltene Zwischentauschmittel eingeführt, wie z.\,B. Kaurischneckenhäuser in China. In der Folge wurden daraus Münzen und schließlich unser heutiges Geld. Das klingt zunächst wie eine perfekte Lösung.

Ein großes Problem ist die Verwaltung des Geldes. Heutzutage wird es vermehrt ohne einen echten Gegenwert zu haben. Durch das Zinssystem gibt es immer mehr Schulden als (virtuelles) Geld, d.\,h. die ganze Welt kann den privatisierten Banken gegenüber insgesamt niemals schuldenfrei sein. Großkonzerne und Banken haben Geld und demnach Macht über jeden, der Geld benötigt. Nicht nur normale Bürger, auch die Regierungen gehören dazu. Folglich bringt es rein garnichts in die Politik zu gehen, wenn man etwas daran ändern möchte -- die sogenannte "`Elite"', wie sie Verschwörungstheoretiker nennen, wird das verhindern. Das ist nur leider keine Verschwörungstheorie sondern eine Tatsache. Die genauen Verhältnisse kann man überall selbst nachlesen.

Ein weiteres entscheidendes Problem ist relativ jung. Wir sind an einem Punkt angelangt, an dem wir die gesamte Welt problemlos versorgen könnten.\footnote{Siehe dazu auch die Ergebnisse des Venus"=Projektes.} Doch alles was man zum Leben benötigt kostet Geld und Geld bekommt man nur durch Arbeit. Wie schafft man sich Arbeit, wenn es eigentlich kaum welche geben muss? Man produziert billig und kurzlebig. Man sorgt künstlich für Knappheit, denn was selten ist, ist wertvoll. Beispielsweise werden in der Kimberly Diamantenmine Diamanten verbrannt, um den Preis hoch zu halten. Darüberhinaus lässt man monotone und anspruchslose Arbeit von Menschen statt von Maschinen erledigen um die Arbeitslosenzahlen niedrig zu halten, denn man braucht ja Arbeit, um zu leben -- selbst wenn sie vollkommen sinnlos ist. Effizienz, Reichhaltigkeit und Nachhaltigkeit widersprechen folglich der Struktur des profitorientierten Systemes.

Das Wohl eines Menschen darf demnach nicht davon abhängen, möglichst viel Profit zu erwirtschaften oder allgemein gesagt Probleme zu lösen, denn sonst erzeugt er sie künstlich (wie Krieg oder Nahrungsmittelvernichtung).

\subsection{Arbeit}\label{sec:situation/work}

Ich führe nun den letzten Gedanken des letzten Abschnittes fort. Man soll demnach nicht arbeiten müssen, um ein Mindestwohlstandslevel zu erhalten. Für Luxusgüter wäre dies dagegen denkbar. Die Frage, die sich dabei auf tut ist folgende: Wie soll die Welt funktionieren, wenn niemand arbeitet? Ein Punkt wurde bereits erwähnt und ist erheblich: Fortschritt und damit auch Automatisierung wird durch Patente oder sinnlose (es gibt auch sinnvolle) Parallel- oder Neuentwicklungen derselben oder bereits existierender Technologien bzw. Produkte erschwert. Oft sind Maschinen in unserem Wirtschaftssystem auch einfach nur unwirtschaftlich; eine Eigenschaft, welche in einem nicht"=monetären System gänzlich anders definiert wäre (die Grenzen wären nicht das Kapital und die eigenen Kompetenzen sondern lediglich die tatsächlich vorhanden Ressourcen und das Ziel wäre nicht die lokale Maximierung von Profit unter Anwendung diverser Tricks und Manipulationen auf Kosten des Gesamtwohles, sondern die direkte und tatsächliche Maximierung des Gesamtwohles.

Ein weiterer Punkt ist die falsche Annahme, dass wir nur für Geld arbeiten. Es gibt genügend andere Motive. Dies beweist beispielsweise die Tatsache, dass sich sogar jetzt schon etwa 10--40\% der Konsumenten meist ohne direkte materielle Gegenleistungen an (Weiter"~)""Entwicklungen, Modifikationen und Verbesserungen von Produkten beteiligen \citep[S.~11]{piller_oi_2006}. Folgende Beweggründe sind dafür verantwortlich \citep[SS.~11--13]{piller_oi_2006}.
\begin{compactitem}
\item Extrinsische Motive (Selbstnutzungserwartung).
\item Intrinsische Motive (Befriedigung des Spieltriebes (Spaß, Kreativität) und des Forschungsdranges verbunden mit dem Erreichen eines Flow"=Zustandes (Fesselung von der Arbeit), sofern eine erfüllbare Herausforderung erkannt wird; eine direkte Leistungsrückkopplung steigert den Effekt).
\item Soziale Motive (Anerkennung der eigenen Arbeit (von anderen Menschen, insbesondere von Mitarbeitern oder einer entsprechenden Community) und gegenseitige Unterstützung).
\end{compactitem}
Das Potenzial wird auch anhand einer Studie deutlich, nach der 6,2\% der britischen Verbraucher innerhalb der beobachteten drei Jahren an Innovationsprozessen beteiligten, d.\,h. entweder eigene Produkte geschaffen oder bestehende angepasst haben. Die jährlichen Aufwendungen übertrafen dabei den Forschungs"= und Entwicklungskosten für Verbrauchsgüter aller britischen Unternehmen auf das 2,3"=fache \citep{hippel_2010}.
%(vgl. auch die Entwicklung von \textit{Closed Innovation} zu \textit{Open Innovation})

\subsection{Gesetze}\label{sec:situation/laws}

Jedes Lebewesen ist im Kern egoistisch, das ist biologisch überlebensnotwendig. Allerdings muss es mit der Umwelt auf gewisse Weise kooperieren, wenn es überleben will\footnote{\textit{Überleben} als obersten Willen darzulegen ist eigentlich falsch, genügt aber hier. Das tatsächliche Lebensziel wird in \myvref{sec:basis/aim} behandelt.}. Folglich muss sich ein Organismus einschränken und kontrollieren, um mit anderen erfolgreich zu interagieren. Diese symbiotischen Beziehungen werden komplexer, je wirkungsvoller und weitreichender (auch zeitlich) sie ausgelegt sind und können von einem Organismus ohne weiteres nicht vollständig überschaut werden. Daher sind (global) einzuhaltende Regeln erforderlich, um letzlich das eigene Wohl sicherzustellen. Aus diesem Grund haben sich auch Grundbedürfnisse und Instinkte ausgebildet und sind wir Rudeltiere geworden: Unsere Vorfahren haben sich in Gruppen überschauberer Größe organisiert. Evolutionstechnisch ist es daher auch nicht abwegig Fremde zunächst als Feinde zu betrachten. Folglich wirkt Altruismus höchstens in einem Familien"=Clan.

Die durch unsere Gesetze und Verhaltensweisen derzeit eher destruktive Beeinflussung der Welt liegt darin begründet, dass wir unsere vererbten Verhaltensweisen dank komplexer Lernfunktionalitäten in immer kürzerer Zeit umgehen können. Das unausweichlich eingeschränkte Gesamtbild, Ignoranz bzw. Egoismus und damit Machtgier sind hierbei grundlegende Faktoren für die Fehlentwicklung. Bisher konnte sich die Natur bei solchen extremen Abweichungen durch genetische Anpassungen immer selbstständig die Waage halten. Seit der Mensch aber so "`vehement viel denkt"', laufen die Veränderungen viel zu schnell und in einem zu großem Maßstab ab, wobei er die Auswirkungen selbst erst jetzt zu überblicken vermag.

\subsection{Demokratie}\label{sec:situation/democracy}

Es wird allgemein behauptet, dass wir in einer Demokratie leben. Tatsächlich haben wir aber kaum Einfluss auf Entscheidungen. Wir wählen Parteien, welche anschließend machen was sie wollen. Wäre dies nicht der Fall und würden sie für das Allgemeinwohl des Volkes arbeiten, würde nicht so viel Elend in der Welt herrschen, das genügt hier als Beweis.

Der Sinn dahinter war, dass Entscheidungen heruntergebrochen werden um die Komplexität zu verringern. Allerdings gehen übergeordnete Instanzen genauso wenig auf untergeordnete ein, wie Parteien auf die Bürger eingehen. Hier und da gibt es zwar Regeln, allerdings werden sie immer weiter aufgeweicht, hauptsächlich mit Hilfe psychologischer Manipulation, wodurch untergeordnete Instanzen, vor allem normale Bürger, diversen Initiativen quasi aus dem falschen Bauchgefühl heraus zustimmen. Dazu zählt insbesondere der Terror"=Hype. Seine Folgen sind beispielsweise die Lockerung des Datenschutzes inkl. steigender Überwachung oder die Aushebelung des deutschen Gesetzgebungsverfahrens durch den Vertrag von Lissabon.

Allzu direkt darf die Demokratie jedoch auch nicht sein. Als einfacher Bürger kann man die Komplexität hinter den Entscheidungen garnicht überblicken und handelt oft voreingenommen. Die Aufgabe der Parteien und Lobbyisten ist es, komplizierte Sachverhalte zu analysieren und Lösungen mit entsprechenden Begründungen vorzuschlagen, welche dann akzeptiert werden können oder nicht. Entscheidungen beruhen also auf dem Vertrauen weniger Instanzen, welche dieses natürlich zu ihren Gunsten ausnutzen können.

\subsection{Revolutionen}\label{sec:situation/revolutions}

Versuche, den von Karl Marx und Friedrich Engels prophezeiten Kommunismus durchzusetzen scheiterten schon an der sozialisitischen Vorstufe und am Wirtschafts"=Management.

Allgemein scheitern Revolutionsversuche, welche keine Diktatur hervorbringen wollen, an der Trägheit der Menschen. Die an den wichtigen Positionen zu wohlhabende und allgemein müde Masse beginnt nicht auf Basis von systemfeindlichen Diskussionen, sich rückhaltlos dem gewöhnlichem Alltag zu widersetzen, geschweige denn kollektiv und ausreichend konsequent. Dies wäre existenzgefährdend. Es müsste schon gewaltiges schief laufen, allerdings ist dies sehr unwahrscheinlich: Man will seine Sklaven schließlich behalten. Die "`Elite"' wird uns füttern, gerade so viel, dass wir arbeiten können und mangels Alternativen auch arbeiten werden.

Den letzten Punkt kann man gut anhand eines Spielkasinos verdeutlichen: Wir sind quasi die Spieler und die Politiker sind maximal die Croupiers. Warum sollten die Besitzer des Kasinos -- Großkonzerne und Banken -- etwas an den Regeln ändern? Oder gar das Kasino für alle freigeben? Es läuft für sie nunmal so wie es läuft.

\subsection{Analyse}\label{sec:situation/analysis}

Wir stellen also fest:
\begin{itemize}
\item Eine profitorientierte oder ressourcen"=arme Gesellschaft erzeugt Gier und damit Machtstreben. Ersteres bedingt der Kapitalismus und letzteres wird für ihn konstruiert.
\item Allgemeingültige Selbstlosigkeit bzw. Solidarität ist nicht angeboren.
\item Absolute Demokratie führt zu Chaos und benachteiligt Randgruppen; abgeschwächte Demokratie führt zum Kontrollverlust.
\item Wir werden in unseren Handlungsmöglichkeiten immer weiter eingeschränkt. Schon jetzt sind wir Sklaven, aber noch arbeiten wir unter freiem Himmel.
\item Wir sind technologisch so gut wie festgefahren. Noch länger auf einen Durchbruch zu warten, welcher uns zu einem besseren System befähigt, ist zu riskant.
\item Die Organisation bzw. der Aufruf zu revolutionären Tätigkeiten darf nicht von einer Minderheit ausgehen
\end{itemize}

Zusammenfassend kann man sagen, dass unser System grundlegende Widersprüche aufweist und Änderungen im benötigen Maßstab in ihm nicht möglich sind; wählen gehen genügt nicht. Man muss ganz von vorne beginnen. Mit den politischen Mitteln, die uns zur Verfügung gestellt werden, kommen wir nicht weiter. Die bloße Verteilung von aufklärenden Informationen genügt aber auch nicht.