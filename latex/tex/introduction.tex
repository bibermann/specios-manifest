\chapter{Einführung}\label{chap:introduction}

Dieses Kapitel wird den Hintergrund und die Notwendigkeit von einem grundlegend neuen Gesellschaftssystem verdeutlichen.

Verschiedenste Probleme sind allgemein bekannt:
\medskip
\begin{compactitem}
\item Hunger, Verschmutzung, niedrige Lebensqualität
\item Tierquälerei (Massentierhaltung mit allen Konsequenzen, Fischerei\footnote{Fische haben ein komplexes Schmerzbewusstsein analog zu Säugetieren, wie Studien der Gehirnaktivität und des Folgeverhaltens belegen. \citep{spiegel_10_2011}}, Pelz"=Beschaffung, überflüssige Tierversuche, Missbrauch für Unterhaltungszwecke)
\item Zerstörung der Natur (Abholzung, Ausrottung, Freisetzung fossiler CO\textsubscript{2}"=Einlagerungen, chemische und atomare Kontaminierung)
\item tödliche Auseinandersetzungen (Kriege, Terror"=Anschläge)
\item niedere Arbeiten
\item Überbevölkerung
\end{compactitem}
\medskip

Ein großer Fehler bei den meisten bisherigen Versuchen einer entsprechenden Problemlösungsstrategie oder zunächst einmal einer Ursachensanalyse innerhalb der Gesellschaftsordnung war der, nicht naturwissenschaftlich an die Probleme heranzugehen; geblendet von Religionen und klassischen Ideologien. Es fängt bei der Argumentation an, was denn das Ziel für die Gesellschaft und den einzelnen Menschen darstellen soll, welche nur auf Grundlage willkürlicher Moralvorstellungen oder überhaupt nicht geführt wurde.
